\InputIfFileExists{../global.src}\relax\relax
\iffull
\title[Organisations Tutorium - Allgemeine Informationen]{Das bin ich\ldots\ Das seid Ihr! Schnabeltier.\\\small Organisations Tutorium}
\date{\sffamily KW 43}
\fi

\newsavebox\pinguA \newsavebox\pinguB
\iffull
\savebox\pinguA{\tikz\pingu[eyes shock, wings shock];}
\savebox\pinguB{\tikz{\pingu[eyes shiny, left wing wave,tie=pingu@purple,tie dots,halo, staff right,hair 3=pingu@purple,name=X]; \node[above right,xshift=1mm] at(X-wing-left) {He... Helloh?};}}

\begin{document}
\begin{frame}[plain,c]
\begin{tikzpicture}[remember picture, overlay]
    \onslide<2->{%
        \node[above left=-3cm,scale=2.25] at(current page.south east) {\rotatebox{45}{\usebox\pinguA}};
    }
    \onslide<3->{
        \node[below right=1cm,scale=2,align=left] at(current page.north west) {\strut Könnt ihr\ldots~~\only<4->{Könnt ihr mich {\bfseries sehn}?}\\\strut\only<5->{Und etwa auch\ldots~~\only<6->{angrabbeln?\llap{\smash{\raisebox{-.35\baselineskip}{\scalebox{.35}{(bitte nich)}~~}}}}}};
    }
    \onslide<7->{%
        \node[above right=1cm] at(current page.south west) {\usebox\pinguB};
    }
\end{tikzpicture}
\end{frame}

\Titlepage{Org}

\section{Allgemeines}
\fi
\savebox\pinguA{\tikz\pingu[right eye wink,right wing wave,bow tie=paletteD,left wing grab,cup=paletteC];}
\savebox\pinguB{\tikz\pingu[eyes angry,left wing hug,bow tie=paletteB];}

\begin{frame}[c]{Hygienekonzept}
    \begin{tikzpicture}[remember picture, overlay]
        \onslide<12->{\node[below left=1.35cm,xshift=.35cm] at(current page.east) {\scalebox{.6}{\usebox\pinguA}};}
        \onslide<13->{\node[above left=.55cm,xshift=-3.5cm] at(current page.east) {\scalebox{.6}{\usebox\pinguB}};}
    \end{tikzpicture}\vspace*{-\baselineskip}\begin{itemize}[<+(1)->]
        \item Wir machen \only<handout:0|2-7>{5G}\only<8->{\cancel{5G} \textbf{3G}} \begin{itemize}
            \item Geimpft
            \item Genesen
            \item Getestet
            \item \only<8->{\color{lightgray}}Genmanipuliert \& Gechipt
            \item \only<8->{\color{lightgray}}Gelangweilt
        \end{itemize}
        \item<9-> Präsenzlehre so lange wie möglich. \onslide<10->{(Sonst \textit{Zoom})}
        \item<11-> Abstand einhalten!
    \end{itemize}
\end{frame}


\begin{frame}[t]{Zu den Folien und Tutorien}
    \begin{itemize}[<+(1)->]
        \itemsep=10pt
        \item Folien von Florian Sihler (\href{mailto:florian.sihler@uni-ulm.de}{florian.sihler@uni-ulm.de}).
        \item Mitschriften, Zusammenfassungen, \ldots{} finden sich im Cloudstore: \begin{itemize}[<1->]
            \item \url{https://cloudstore.uni-ulm.de/apps/circles/}
            \item Kreis: \say{Mitschriebe Informatik} (ganz eingeben)
        \end{itemize}
        \item Bei erbrachter Leistung: 3 ECTS (LP)
    \end{itemize}
\end{frame}


\savebox\pinguA{\tikz\pingu[left wing hug,eyes wink];}
\savebox\pinguB{\tikz\pingu[right wing hug,eyes shiny];}

\begin{frame}[t]{Übungsblätter\hfill I}
\begin{itemize}[<+(1)->]
    \item Mindestens 50\,\% der Punkte pro Blatt
    \item Teilnahme am Tutorium verpflichtend für Punkte
    \item Maximal zwei Blätter nicht bestanden
    \begin{itemize}
        \item Bonusaufgaben auf sechs Blättern
        \item Absolviert man mindestens die Hälfte $\rightarrow$ Ein Pufferblatt mehr
    \end{itemize}
    \item Abgabe in zweier Gruppen \only<handout:0|7>{\smash{\raisebox{-2pt}{\resizebox*!{\baselineskip}{\usebox\pinguA\!\usebox\pinguB}}}}\only<8->{\smash{\raisebox{-2pt}{\resizebox*!{\baselineskip}{\usebox\pinguA\hskip8em\usebox\pinguB}}}}
    \item<9-> Erlaubte Formate: \begin{itemize}
        \item<10-> Bei einer Datei: PDF, Java-Quellcode
        \item<11-> Bei mehreren: Archiv verwenden! (ZIP,~\ldots)% TODO: NOTE: wird vermutlich noch Freigeschaltet
    \end{itemize}
    \item<12-> Programme \emph{müssen} lauffähig sein (inklusive \bjava{main}).\\[-.2\baselineskip]
    \onslide<13->{\info{Oder entsprechend kommentiert, wo es (wieso) Probleme gibt.}}
\end{itemize}
\end{frame}

\begin{frame}[t]{Übungsblätter\hfill II}
    \begin{itemize}[<+(1)->]
        \item \emph{Jeder} muss die Abgabe vorstellen können.\pause\\[-.2\baselineskip]\info{Unabhängig davon, wer die Aufgabe im Team gemacht hat.}
        \item Im Falle einer Krankheit:\pause\\\emph{Nachricht an den Tutor mit zugehöriger Entschuldigung!}
    \end{itemize}
    \vfill
    \begin{center}
        \onslide<6->{\bfseries Lest die \href{https://www.uni-ulm.de/einrichtungen/zuv/dez1/recht-und-organisation/satzungen-und-ordnungen/studium-promotion-habilitation/studien-und-pruefungsordungen-bachelor-master-staatsexamen/}{\textbf{FSPO}}!\\}
        \onslide<7->{\bfseries Verwendet die Uni-Mail!\\}
    \end{center}
\end{frame}

\iffull
\begin{frame}[t]{Präsenzaufgaben}
    \begin{itemize}[<+(1)->]
        \itemsep10pt
        \item Stift und Papier
        \item Jede Woche, unbepunktet
        \item Darf aber gern abgegeben werden!
    \end{itemize}
\end{frame}


\immediate\write18{wget https://media.githubusercontent.com/media/EagleoutIce/Episode-Heaps/gh-pages/preview-01.png -O logoHeaps-\jobname.png}
\immediate\write18{wget https://media.githubusercontent.com/media/EagleoutIce/Episode-Recursion/gh-pages/preview-01.png -O logoRecursion-\jobname.png}
\immediate\write18{wget https://media.githubusercontent.com/media/EagleoutIce/Episode-Traversierung/gh-pages/preview-01.png -O logoTraversal-\jobname.png}
% and more :D

\begin{frame}[t]{Zusätzliches}
    \begin{itemize}[<+(1)->]
        \itemsep9pt
        \item Es empfiehlt sich die Abgaben in \LaTeX\ ($2_{\textstyle\varepsilon}$) zu machen.
        \item Dafür wurden erklärende Bonusblöcke erstellt. Darunter:\pause{}
        \vspace*{-.75\baselineskip}\begin{multicols}{3}
            \begin{enumerate}[<1->]
                \item \LaTeX{} (2-3 Tutorien)
                \item Richtig Kommentieren
                \item Linux (\& Kommandozeile)
                \item Git -- Versionsverwaltung
                \item (Unit-)Testen
                \item \ldots\par
            \end{enumerate}
        \end{multicols}\vspace*{-.75\baselineskip}
        \item Mit dem letzten Semester gibt es auch tolle Episoden. Darunter:\\\smallskip
% hella hail the copy and paste
\resizebox{.85\linewidth}!{\begin{tikzpicture}[align-base]
    \onslide<2->{\draw[thick,darkgray,rounded corners=2.5pt,path picture={\node at(path picture bounding box.center) {\href{https://media.githubusercontent.com/media/EagleoutIce/Episode-Traversierung/gh-pages/noanim_traversal.pdf}{\includegraphics[width=8.5cm,height=4.788cm,keepaspectratio]{logoTraversal-\jobname.png}}};}] (0,0) rectangle (8.5cm,4.788cm);}
\end{tikzpicture}\quad\begin{tikzpicture}[align-base]
    \onslide<2->{\draw[thick,darkgray,rounded corners=2.5pt,path picture={\node at(path picture bounding box.center) {\href{https://media.githubusercontent.com/media/EagleoutIce/Episode-Recursion/gh-pages/noanim_rekursion.pdf}{\includegraphics[width=8.5cm,height=4.788cm,keepaspectratio]{logoRecursion-\jobname.png}}};}] (0,0) rectangle (8.5cm,4.788cm);}
\end{tikzpicture}\quad\begin{tikzpicture}[align-base]
    \onslide<2->{\draw[thick,darkgray,rounded corners=2.5pt,path picture={\node at(path picture bounding box.center) {\href{https://media.githubusercontent.com/media/EagleoutIce/Episode-Heaps/gh-pages/noanim_heap.pdf}{\includegraphics[width=8.5cm,height=4.788cm,keepaspectratio]{logoHeaps-\jobname.png}}};}] (0,0) rectangle (8.5cm,4.788cm);}
\end{tikzpicture}\qquad\textbullet~\textbullet~\textbullet}
        \item Die (Lösungs-)Folien werden wochenweise veröffentlicht.
    \end{itemize}
\end{frame}

\section{Java \& Editoren}
\def\os#1{\;\textsuperscript{\tiny\color{black}\def\x##1{\textcolor{gray}{##1}}#1}}
\begin{frame}[t]{How to Editor}
    \begin{itemize}[<+(1)->]
        \itemsep10pt
        \item Text-Editor mit Syntaxfreude
        \item Keine IDE!
        \item Klassisch: \begin{itemize}
            \item \href{https://notepad-plus-plus.org/}{Nodepad++}\os{\faWindows, \x{\faApple}, \x{\faLinux}}
            \item \href{https://code.visualstudio.com/Download}{VS-Code}\os{\faApple, \faWindows, \faLinux} (ohne Extensions)
            \item \href{https://atom.io/}{\textsc{atom}}\os{\faApple, \faWindows, \faLinux}
        \end{itemize}
        \item Advanced: \begin{itemize}
            \item \href{https://www.nano-editor.org/}{nano}\os{\faLinux, \faApple, \x{\faWindows}}
            \item \href{https://www.vim.org/}{vim}\os{\faLinux, \faApple, \x{\faWindows}}
        \end{itemize}
    \end{itemize}
\end{frame}

\begin{frame}[t]{How to Java}
    \begin{itemize}[<+(1)->]
        \itemsep7pt
        \item Linux: \begin{itemize}
            \item Wenn \T{apt}-basiert: \cbash{sudo apt install default-jdk} (\bbash{openjdk-14-jdk},~\ldots)
            \item Auch sonst über den jeweiligen Paketmanager
            \item Oder per \href{https://www.oracle.com/java/technologies/javase-downloads.html}{download}.
        \end{itemize}
        \item Windows: \begin{itemize}
            \item Download \& Installation der \href{https://www.oracle.com/java/technologies/javase-downloads.html}{gewünschten Version}.
            \item \T{PATH}-Variable anpassen! Dafür gibt es etliche \href{https://explainjava.com/set-java-path-and-java-home-windows/}{Anleitungen}:
                \begin{enumerate}
                    \item \say{Systemumgebungsvariablen bearbeiten}\textor \say{Edit the system environment variables}
                    \item \say{bin}-Ordner der JDK zum \T{PATH} hinzufügen. \info{Zum Beispiel: \say{C:\textbackslash Program Files\textbackslash Java\textbackslash jdk-15\textbackslash bin}}
                \end{enumerate}
        \end{itemize}
        \item MacOS: \begin{itemize}
            \item Download \& Installation der \href{https://www.oracle.com/java/technologies/javase-downloads.html}{gewünschten Version}.
            \item Sonst unterstützt auch weitere \href{https://brew.sh/}{Homebrew} ein paar Versionen.
        \end{itemize}
        \item Es gibt auch \href{https://jdk.java.net/16/}{open source} Implementationen.
    \end{itemize}
\end{frame}


\section{End-Pinguuu}
\begin{frame}[c]
\null\vfill
\centering\begin{tikzpicture}[scale=1.5,remember picture]
    \pingu[wings wave,name=saphira,eyes wink,pants=cprimary,monocle right]
    \onslide<2->{\path[postaction={decorate},decoration={text along path, text={|\huge\Fontauri|Motivation!},text align={fit to path}}] (saphira-bill) ++ (90+55:45pt) arc (90+55:90-55:45pt);}
\end{tikzpicture}
\end{frame}

{\setbeamercolor{background canvas}{bg=cprimary!15!black}
\begin{frame}[plain,c]
\null\vfill
\centering\begin{tikzpicture}[scale=1.5,remember picture]
    \pingu[wings raise,monocle right,name=saphira,eyes wink,devil horns,left eye devil,right eye angry,bill=angry,glow thick=white,heart=pingu@purple!60!black]

    \path[postaction={decorate},decoration={text along path, text={|\huge\Fontauri\bfseries\color{white}|Motivation!},text align={fit to path}}] (saphira-bill) ++ (90+55:45pt) arc (90+55:90-55:45pt);
\end{tikzpicture}
\end{frame}}
\end{document}
\else % shorter pengus
\section{End Pinguuu}
{\begin{frame}[plain,c]
\null\vfill
\centering\begin{tikzpicture}[scale=1.5,remember picture,overlay]

    \scope[shift=(current page.center)]
    \pingu[wings wave,name=saphira,eyes wink,pants=cprimary,monocle right,xshift=-3.25cm]
    \path[postaction={decorate},decoration={text along path, text={|\huge\Fontauri|Motivation!},text align={fit to path}}] (saphira-bill) ++ (90+55:45pt) arc (90+55:90-55:45pt);

    \pingu[wings raise,monocle right,name=saphira,eyes wink,devil horns,left eye devil,right eye angry,bill=angry,glow thick=white,heart=pingu@purple!60!black,xshift=2cm]

    \path[postaction={decorate},decoration={text along path, text={|\huge\Fontauri\bfseries|Motivation!},text align={fit to path}}] (saphira-bill) ++ (90+55:45pt) arc (90+55:90-55:45pt);
    \endscope
\end{tikzpicture}
\end{frame}}
\fi
\RequirePackage[cmintegrals,cmbraces,ebgaramond]{newtxmath}
\InputIfFileExists{../global.src}\relax\relax
\iffull
\usepackage{foekfont,addfont,punk,bbold,inslrmin}
\usepackage{xfrac}

\input RoyalIn.fd
\newcommand*\ROSEBABY{\usefont{U}{RoyalIn}{xl}{n}}

\title[Erstes Tutorium -- Übungsblatt 1]{Mathe. Haha. Kann ich.\\\small Tutorium Unostasio}
\date{\sffamily KW 44}

\begin{document}

\Titlepage{1}
% space pingu stage
\begin{frame}[c,plain]
    \centering
    \onslide<5->{\begin{tikzpicture}[align-bottom]%
        \pingu[body=pingu@main!90!white,cloak=black!99!darkgray,sunglasses,glasses opacity=1,normal,heart=black!15!white,bill=angry]
    \end{tikzpicture}}~~~\onslide<6->{\begin{tikzpicture}[align-bottom]%
        \pingu[body=pingu@main!90!white,name=a,sunglasses round,glasses round opacity=1,small,heart=cprimary!15!white,wings raise,halo,bow tie=cprimary,halo raise=1mm,feet simple]
        \onslide<7->{\node[above=.75cm,align=center] at (a-head-top) {All hail the space!\\[-.65mm]\only<8->{\scriptsize \makebox[0pt]{Lehrfeld? Lärfeld? Learfeld? Nix?}}};}
    \end{tikzpicture}}~~~\onslide<5->{\begin{tikzpicture}[align-bottom]%
        \pingu[body=pingu@main!95!white,cloak=black!98!darkgray,sunglasses,glasses opacity=1,normal,heart=black!17!white,bill=angry]
    \end{tikzpicture}}
    \par\onslide<2->{\begin{tikzpicture}[align-bottom]%
        \pingu[:back,hair 1=cprimary,cane right=pingu@bronze!80!brown,cane right raise=3mm,name=a,left wing raise]
        \node[white,align=center,yshift=-2mm] at(a-bill-bottom) {\foekfamily SPACE\\{\punkfamily LOVE}};
    \end{tikzpicture}}~\onslide<4->{\begin{tikzpicture}[align-bottom]
        \pingu[:back,small,body=pingu@main!95!cprimary,hair 2=cprimary,name=a,wings wave,lollipop left=cprimary,feet color=pingu@yellow!95!brown]
        \node[white,align=center,yshift=-3mm,scale=3] at(a-bill-bottom) {\ROSEBABY S};
    \end{tikzpicture}\hskip-1.9em\begin{tikzpicture}[align-bottom]
        \pingu[:back,large,body=pingu@main!86!brown,hair 3=cprimary,hat,name=a,left wing hug,feet color=pingu@yellow!90!brown]
        % \node[white,align=center,yshift=-3mm,scale=3] at(a-bill-bottom) {\ROSEBABY S};
    \end{tikzpicture}}~~\onslide<3->{\begin{tikzpicture}[align-bottom]
        \pingu[:back,body=pingu@main!93!black,hair 4=cprimary,name=a,right wing raise,flag right=darkgray!80!cprimary!50!black,right item flip]
        \node[white,align=center,yshift=-1mm] (cl) at(a-bill-bottom) {\bbfamily Calling};
        \path[postaction={decorate},decoration={text along path, text={|\fontsize{4pt}{4pt}\selectfont\iminfamily|straight from},text align={fit to path},text color=white}] ([yshift=2mm]cl.south) ++ (270-75:22pt and 10pt) arc (270-75:270+75:22pt and 10pt);
        \node[white,below=1.9mm,scale=.8,align=center] at(cl.south) {\def\rmdefault{AlphaSlabOne-TLF}\rmfamily outta\\[-3mm]\scriptsize\T{\textvisiblespace}};
        % get space up there baby
        \node[white,xshift=7mm,yshift=10mm,xscale=1.33] at(a-wing-right-tip.north) {\bjava[showspaces]{\ }};
    \end{tikzpicture}}\hspace*{-1em}\null
\end{frame}
\fi

\section{Präsenzaufgabe}
\begin{frame}{Präsenzaufgabe}
    \begin{aufgabe}{Kommen Sie hier rein?}
        \pause Konstruieren Sie für jede der folgenden Aussagen einen booleschen Ausdruck, welcher diese überprüft.
        \begin{enumerate}[<+(1)->]
            \item Eine Person ist Teenager. (\bjava{int alter} in Jahren)
            \item Es ist Vormittag. (\bjava{int uhrzeit} im 24h-Format)
            \item Eine Tür ist geschlossen. (\bjava{boolean istOffen})
            \item Im Kaffee ist entweder Zucker \emph{oder} Milch, aber nicht beides. (\bjava{boolean zucker} und \bjava{boolean milch})
        \end{enumerate}
        \onslide<1->
    \end{aufgabe}
\end{frame}

\def\spar{\par}
\begin{frame}[t,fragile]{Präsenzaufgabe - Lösung}
    \begin{itemize}[<+(1)->]
        \itemsep=\medskipamount% TODO: talk about teenager problem
        \item Teenager: \pause\bjava{alter >= 13 && alter <= 18} \info{oooder 19?}
        \item Vormittag: \pause\bjava{uhrzeit >= 9 && uhrzeit <= 12}
        \item Tür: \pause\bjava{!istOffen} \info{But who trusts the name?}
        \item (Guter?) Kaffee: \pause\T{(zucker || milch) \pause\&\& !(zucker \&\& milch)}, oder: \pause\bjava{zucker ^ milch}.
    \end{itemize}\bigskip
  \begin{columns}
    \column{0.75\linewidth}\pause{\textit{\bvoid{^} entspricht dem XOR-Operator.
     \onslide<17->{Dieser evaluiert nur genau dann zu wahr, \onslide<18->{wenn einer der Parameter wahr und der andere falsch ist.}\spar}\vspace*{1.5cm}}}
    \column{0.25\linewidth}{\onslide<12->{\vspace*{\dimexpr-\topsep-2\baselineskip}\begin{table}[H]\scriptsize
        \centering\begin{mtabular}{ccc}
            A & B & A \^{} B \\
            \onslide<13->{0 & 0 & 0
            \onslide<14->{\\0 & 1 & 1
            \onslide<15->{\\1 & 0 & 1
            \onslide<16->{\\1 & 1 & 0}}}}
        \end{mtabular}
    \end{table}}}
    \end{columns}\vspace*{-0.25cm}
\end{frame}


\def\h#1{\textcolor{cprimary}{#1}}
\def\brace#1{(#1)}
\section{Übungsblatt 1}
\subsection{Aufgabe 1}
\begin{frame}{Übungsblatt 1 - Aufgabe 1 a)}
    \begin{itemize}[]
        \item \pause Hornerschema am Beispiel von $1101_{(2)}$: \pause\[(((\h{1}\cdot2 + \h{1})\cdot2 + \h{0}) \cdot 2) + \h{1} \pause = (((3)\cdot2 + \h{0}) \cdot 2) + \h{1} \pause = (6 \cdot 2) + \h{1} = 13_{(10)}.\vspace*{-0.5cm}\]
        \item \pause \(1010101_{(2)} \pause = \brace{
            \brace{
                \brace{
                    \brace{
                        \brace{
                            \h{1} \cdot 2 + \h{0}
                        } \cdot 2 + \h{1}
                    } \cdot 2 + \h{0}
                } \cdot 2 + \h{1}
            } \cdot 2 + \h{0}
        } \cdot 2 + \h{1} \pause = 85_{(10)}\)
        \item \pause \(12345_{(8)} \pause =
            \brace{
                \brace{
                    \brace{
                        \brace{
                            \h{1} \cdot 8 + \h{2}
                        } \cdot 8 + \h{3}
                    } \cdot 8 + \h{4}
                } \cdot 8 + \h{5}
            }
        \pause = 5349_{(10)}\)
        \item \pause Hexadezimal:\pause{} \vspace*{-0.75\baselineskip}\begin{multicols}{4}
                  \begin{enumerate}[]
                      \foreach[count=\i] \a in {1,...,9,A,B,...,F}{
                      \item $\a_{(16)}\; \widehat{=}\; \i_{(10)}$
                            }
                      \item %place holder for NOPE-balancing :D
                  \end{enumerate}\vspace*{-\baselineskip}
              \end{multicols}
        \item \pause \(FACE_{(16)} \pause = ((\h{15} \cdot 16 + \h{10})\cdot 16 + \h{12}) \cdot 16 + \h{14} \pause = 64206_{(10)}\)
    \end{itemize}
\end{frame}

\begin{frame}{Übungsblatt 1 - Aufgabe 1 a)}
    \begin{itemize}[<+(1)->]
        \item Wir können die Rechnungen auch tabellarisch veranschaulichen!\pause{} Zur Erinnerung:\pause{} \(1010101_{(2)} = \brace{
            \brace{
                \brace{
                    \brace{
                        \brace{
                            \h{1} \cdot 2 + \h{0}
                        } \cdot 2 + \h{1}
                    } \cdot 2 + \h{0}
                } \cdot 2 + \h{1}
            } \cdot 2 + \h{0}
        } \cdot 2 + \h{1} = 85_{(10)}\).
    \end{itemize}
    \TikzHorner[4]{1,0,1,0,1,0,1}{2}
\end{frame}

\begin{frame}{Übungsblatt 1 - Aufgabe 1 a)}
    \begin{itemize}[<+(1)->]
        \item Analog:\pause{} \(12345_{(8)} =
        \brace{
            \brace{
                \brace{
                    \brace{
                        \h{1} \cdot 8 + \h{2}
                    } \cdot 8 + \h{3}
                } \cdot 8 + \h{4}
            } \cdot 8 + \h{5}
        }
     = 5349_{(10)}\).\pause{}
    \end{itemize}
    \TikzHorner[3]{1,2,3,4,5}{8}
\end{frame}

\begin{frame}{Übungsblatt 1 - Aufgabe 1 a)}
    \begin{itemize}[<+(1)->]
        \item Sowie:\pause{} \(FACE_{(16)} = ((\h{15} \cdot 16 + \h{10})\cdot 16 + \h{12}) \cdot 16 + \h{14} = 64206_{(10)}\).\pause{}
    \end{itemize}
    \TikzHorner[3]{F,A,C,E}{16}
\end{frame}

\begin{frame}{Übungsblatt 1 - Aufgabe 1 b)\hfill Vorbemerkung}
    \begin{itemize}[]
        \itemsep5pt
        \item \pause$14_{(10)}$ zur Basis 2. Eine Vorbemerkung: \begin{alignat*}{3}
                  \onslide<3->{14 & \div 2 = 7 \qquad & 14 \bmod 2 & = 0\onslide<7->{~~(\leftarrow LSB)} \\}
                  \onslide<4->{7  & \div 2 = 3 \qquad & 7 \bmod 2  & = 1 \\}
                  \onslide<5->{3  & \div 2 = 1 \qquad & 3 \bmod 2  & = 1 \\}
                  \onslide<6->{1  & \div 2 = 0 \qquad & 1 \bmod 2  & = 1\onslide<7->{~~(\leftarrow MSB)}}
              \end{alignat*}
        \item \onslide<8->{Damit gilt: $1110_{(2)} = (((\h{1} \cdot 2 + \h{1})\cdot 2 + \h{1})\cdot 2 + \h{0} \onslide<9->{= 14_{(10)}.}$}
        \item \onslide<10->{\textit{Hinweis: Für \(b2\) genügt der Test auf gerade oder ungerade}.}
        \item \onslide<11->{Dieses Art der Darstellung ist aber \textit{nicht} das Hornerschema!}
    \end{itemize}
\end{frame}

\begin{frame}{Übungsblatt 1 - Aufgabe 1 b)}
    \begin{itemize}[]
        \item Natürlich geht dies auch tabellarisch.\pause{} Zur Erinnerung:\pause{} $1110_{(2)} = (((\h{1} \cdot 2 + \h{1})\cdot 2 + \h{1})\cdot 2 + \h{0} = 14_{(10)}.$\pause{}
              \TikzHornerReverse[4]{1,1,1,0}{2}
    \end{itemize}
\end{frame}

\subsection{Aufgabe 2}
% compensate for fat overflow
\newcommand<>\hlkw[1]{{\savebox0{#1}\only#2{\bfseries\color{paletteA}}\makebox[\dimexpr\wd0+.285ex]{\strut#1}}}
\newcommand<>\hloutkw[1]{{\savebox0{#1}\only#2{\bfseries\color{paletteB}}\makebox[\dimexpr\wd0+.285ex]{\strut#1}}}
\def\g#1{{#1}}\def\k#1{{\savebox0{#1}\makebox[\dimexpr\wd0+.285ex]{\strut#1}}}
\iffull
\begin{frame}[c]{Übungsblatt 1 - Aufgabe 2}
    \pause\centering \parbox{.65\linewidth}{%
    \footnotesize\only<3->{\color{gray}}Angenommen, Sie überlegen sich ein \g{neues Tablet} für die \g{Uni} zu kaufen.
    Sie stellen jedoch fest, dass die \g{Preise für elektronische Geräte} durch den \g{Chipmangel} in \g{letzter Zeit} stark gestiegen sind.
    Nun haben Sie eine \g{Website} gefunden, die Ihnen die jeweiligen \g{\k{Tagespreise}} eines Tablets über die \g{letzten Monate} liefert.
    Sie möchten nun den \g{\k{maximalen pro-}} \g{\k{zentualen Preisanstieg}} zwischen \g{\k{zwei aufeinanderfolgenden Tagen}} für dieses Tablet herausfinden.\bigskip\par
    Entwickeln Sie einen \g{Algorithmus}, der aus einer \hlkw<4->{\g{\k{Liste von Tagespreisen}}} des Tablets den \hloutkw<4->{maximalen prozentualen Preisanstieg} von \hloutkw<4->{\g{\k{zwei aufein-}}} \hloutkw<4->{\g{\k{anderfolgenden Tagen}}} berechnet.}\only<5->{\qquad\parbox{.2\linewidth}{%
        \centering\footnotesize \onslide<6->{\paletteA{Liste von Tagespreise}\\{\color{gray}\((p_1,\;\ldots,\;p_n)_{n\in\N}\)}\medskip\par
        \onslide<7->{\faCaretDown\medskip\par \paletteC{Max. \% Preisanstieg}\\{\color{gray}\(\DeltaMax\)}}}
    }}
\end{frame}

{\def\g#1{{\only<2->{\color{black}}#1}}\def\k#1{{\savebox0{#1}\only<3->{\bfseries\color{paletteA}}\makebox[\dimexpr\wd0+.285ex]{\strut#1}}}
\renewcommand<>\hlkw[1]{{\savebox0{#1}\makebox[\dimexpr\wd0+.285ex]{\strut#1}}}
\renewcommand<>\hloutkw[1]{{\savebox0{#1}\makebox[\dimexpr\wd0+.285ex]{\strut#1}}}
\begin{frame}[c]{Übungsblatt 1 - Aufgabe 2}
    \centering \parbox{.65\linewidth}{\color{gray}\footnotesize Angenommen, Sie überlegen sich ein \g{neues Tablet} für die \g{Uni} zu kaufen.
    Sie stellen jedoch fest, dass die \g{Preise für elektronische Geräte} durch den \g{Chipmangel} in \g{letzter Zeit} stark gestiegen sind.
    Nun haben Sie eine \g{Website} gefunden, die Ihnen die jeweiligen \g{\k{Tagespreise}} eines Tablets über die \g{letzten Monate} liefert.
    Sie möchten nun den \g{\k{maximalen pro-}} \g{\k{zentualen Preisanstieg}} zwischen \g{\k{zwei aufeinanderfolgenden Tagen}} für dieses Tablet herausfinden.\bigskip\par
    Entwickeln Sie einen \g{Algorithmus}, der aus einer \hlkw<4->{\g{\k{Liste von Tagespreisen}}} des Tablets den \hloutkw<4->{maximalen prozentualen Preisanstieg} von \hloutkw<4->{\g{\k{zwei aufein-}}} \hloutkw<4->{\g{\k{anderfolgenden Tagen}}} berechnet.}\qquad\parbox{.2\linewidth} Preisanstieg}\smallskip\\
    }
\end{frame}
}
\fi
\begin{frame}{Übungsblatt 1 - Aufgabe 2 a) \& b)}
    \begin{enumerate}[a)]
        \itemsep8pt
        \item\pause \strut Problemspezifikation: \begin{itemize}[<+(1)->]
            \itemsep3pt
            \item \emph{Tagespreis:} positive reelle Zahl. Dargestellt mit zwei Nachkommastellen.
            \item \emph{Tagespreisliste:} chronologisch sortierte Liste an Tagespreisen \([p_1, \ldots, p_n]\) mit \(n \in \N_1\).
            \item \emph{Tag:} natürlicher Index in Tagespreisliste.
            \item \emph{Zwei aufeinanderfolgende Tage:} zwei aufeinanderfolgende Listenindizes: \(i\) und \(i + 1\) wobei \(i, i + 1 < n\).
            \item \emph{Preisanstieg:} positive Veränderung einer reellen Zahl (dem Tagespreis): \(p_{i + 1} - p_i > 0\).
            \item \emph{maximaler prozentualer Preisanstieg:} größter relativer Preisanstieg: \(\max_{1 \leq i < n} \frac{p_{i + 1} - p_i}{p_i} \cdot 100\).
        \end{itemize}\pause
        {\footnotesize\textbf{Wichtig:} Wir sind hier nicht in einer Programmiersprache: Typen wie \say{\T{int}} gibt es nicht $\Rightarrow$ mathematische Notation benutzen oder eigenen Typ definieren (wie Tupel, \ldots)!}
        \item\pause Problemabstraktion: \begin{itemize}[<+(1)->]
            \itemsep3pt
            \item \emph{Gegeben:} Chronologisch sortierte Liste \([p_1, \ldots, p_n]\) positiver reeller Zahlen. Annahme: \(n > 1\) (\say{letzte Monate}).
            \item \emph{Gesucht:} Positive reelle Zahl \(m = 100\cdot \max_{1 \leq i < n} \frac{p_{i + 1} - p_i}{p_i}\) (\(X\,\%~\widehat{=}~\frac{X}{100}\)).
        \end{itemize}
    \end{enumerate}
\end{frame}

\begin{frame}[t]{Übungsblatt 1 - Aufgabe 2 c) \& d)}
    \begin{enumerate}[a)]
        \setcounter{enumi}{2}
        \item \strut Algorithmenentwurf:
        \setbeamercolor{itemize/enumerate subbody}{fg=black}%
        \begin{enumerate}[(1)]
            \itemsep3pt
            \item\pause Setze $\DeltaMax = 0$.
            \item\pause Setze $i = 1$.
            \item\label{alg:st}\pause Solange $(i < n)$, wiederhole:\smallskip\\
                \pause \quad Setze \(test = \sfrac{(p_{i + 1} - p_i)}{p_i}\).\smallskip\\
                \pause \quad Wenn \(test > \DeltaMax\):~~Setze \(\DeltaMax = test\).\smallskip\\
                \pause \quad Erhöhe \(i\) um \(1\).
            \item\pause Ergebnis ist $100 \cdot \DeltaMax$.
        \end{enumerate}
        \medskip% Wir können Optimieren: einfach von oben herab
        \item\pause Korrektheitsnachweis, Verifikation:\begin{enumerate}[1.]
                  \item \pause Formale vollständige Induktion, oder:
                  \item \pause\say{Textbasiert}.\pause{} \(i\) wächst streng monoton an, die Schleife wird damit genau \(n - 1\) mal durchlaufen.\pause{} Alle mathematischen Berechnungen terminieren per Konstruktion, ebenso die Zuweisungen. Der Algorithmus \textit{terminiert}.\medskip\pause

                  Zudem ist er \textit{partiell korrekt}. Die maximale prozentuale Änderung ist immer größte relative Differenz für alle \([p_1, \ldots, p_{i + 1}]\) im \(i\)-ten Durchlauf.\pause{} Nach \(n - 1\) Durchläufen: \([p_1, \ldots, p_n]\). Basisfall mit \(i = 2\), für Schritt \(i \to i + 1\).
              \end{enumerate}
    \end{enumerate}
\end{frame}

\begin{frame}{Übungsblatt 1 - Aufgabe 2\hfill Notizen}
    \begin{itemize}[<+(1)->]
        \item Totale Korrektheit erfordert zwei Komponenten! \begin{enumerate}
            \item \textit{Terminiertheit}: Der Algorithmus terminiert für jede definierte Eingabe.
            \item \textit{Partielle Korrektheit}: Der Algorithmus liefert für jede definierte Eingabe ein korrektes Ergebnis, sofern er terminiert.
        \end{enumerate}
        \medskip
        \item Es reicht in der Regel nicht aus: \begin{itemize}
            \item \say{Maximum ist sicher größer als alle betrachteten Alternativen}.\pause{} Es muss dann zum Beispiel auch gezeigt werden, dass alle relevanten Alternativen in Betracht gezogen werden.
            \item \say{Der Algorithmus findet das gesuchte Ergebnis}.\pause{} Wir Beweisen zwar (noch) nicht ganz formal. Dennoch sollte ein ausreichendes Verständnis für den Beweis gezeigt werden.
        \end{itemize}\medskip
        \item Vollständige Induktion über \(i\) (meist bei Schleifen): \begin{itemize}
            \item Achtet darauf klar anzugeben, über was die Induktion läuft\pause\ \info{Länge der Einladungs-, Bekannten- oder Bedingungsliste? Das Durchschnittsalter der Person?}
            \item Achtet auf eine vollständige Behauptung. Ist die unvollständig, ist es auch der Prozess an sich.
            \item Induktionsanfang: Wir zeigen, dass es für ein \(i \in \N\) gilt, in der Regel \(i = 0\) oder \(i = 1\).
            \item Induktionshypothese: Die Behauptung gilt für (ein) \(n\).
            \item Induktionsschritt: Gilt die Behauptung für \(n\), so gilt sie auch für \(n + 1\).\pause{} \info{Hier wird fast immer die Hypothese benutzt um den Fall \(n + 1\) auf \(n\) zurückzuführen.}
        \end{itemize}
    \end{itemize}
\end{frame}


\begin{frame}{Übungsblatt 1 - Aufgabe 2 d)\hfill Induktion}
\parallelcontent[c]{\tiny
\setbeamercolor{itemize/enumerate subbody}{fg=black}%
\vspace*{-\baselineskip}\begin{enumerate}[(1)]
    \itemsep1.5pt
    \item\label{algA:a} Setze $\DeltaMax = 0$.
    \item\label{algA:b} Setze $i = 1$.
    \item\label{algA:c} Solange $(i < n)$, wiederhole:\smallskip\\
            \quad Setze \(test = \sfrac{(p_{i + 1} - p_i)}{p_i}\).\smallskip\\
            \quad Wenn \(test > \DeltaMax\):~~Setze \(\DeltaMax = test\).\smallskip\\
            \quad Erhöhe \(i\) um \(1\).
    \item\label{algA:d} Ergebnis ist $100 \cdot \DeltaMax$.
\end{enumerate}
}{\onslide<2->{Wir zeigen die partielle Korrektheit und nehmen die Termination hier als gegeben.}}\medskip\small
    \onslide<3->{Für ein beliebiges aber festes \(n \in \N\) mit \(n \geq 2\) und der Eingabe \(p_1, \ldots, p_n\)} \onslide<4->{zeigen wir per vollständiger Induktion über \(i\),} \onslide<5->{dass in jedem Schritte gelte: \(\DeltaMax \geq \max_{1 \leq j \leq i}(\sfrac{(p_{j + 1} - p_j)}{p_j})\).}\footnotesize
    \begin{description}[xIA:]% Inudktionshypothese als Voraussetzung und Behauptung
        \itemsep0pt
        \item<6->[IA:] Mit \(i = 1\) ist \(\DeltaMax = \frac{p_2 - p_1}{p_1}\) das triviale Maximum (\(1 \leq j \leq 1\)).
        \item<7->[IH:] Es gilt \(\DeltaMax \geq \max_{1 \leq j \leq i}(\sfrac{(p_{j + 1} - p_j)}{p_j})\) für \(i\).
        \item<8->[IS:] Wir zeigen, dass auch \(\DeltaMax \geq\allowbreak \max_{1 \leq j \leq i + 1}(\sfrac{(p_{j + 1} - p_j)}{p_j})\) gilt.
        \onslide<9->{Das heißt, aktuell ist \(\DeltaMax \geq \max_{1 \leq j \leq i}(\sfrac{(p_{j + 1} - p_j)}{p_j})\) und in der weiteren Schleife gilt:}\begingroup\scriptsize
        \begin{enumerate}
            \item<10-> \color{gray}\(test > \DeltaMax\): Durch die Bedingung wird \(\DeltaMax\) aktualisiert. Es gilt: \(\DeltaMax = test \geq \max_{1 \leq j \leq i + 1}(\sfrac{(p_{j + 1} - p_j)}{p_j})\).
            \item<11-> \color{gray}\(test \leq \DeltaMax\): Das neue Glied durch \(i + 1\) ist kein neues Maximum. Es gilt: \(\DeltaMax \geq \max_{1 \leq j \leq i + 1}(\sfrac{(p_{j + 1} - p_j)}{p_j}) \geq test\).
        \end{enumerate}\endgroup
    \end{description}
    \onslide<12->{Nach \(n-1\) Schritten durch die Begrenzung \(i < n\), wurden alle \(p_1, \ldots, p_n\) betrachtet.} \onslide<13->{\(\DeltaMax\) ist Maximum aller.}
\end{frame}

\begin{frame}{Übungsblatt 1 - Aufgabe 2 e)}
    {}{\tiny
        \setbeamercolor{itemize/enumerate subbody}{fg=black}%
        \vspace*{-.9\baselineskip}\begin{enumerate}[(1)]
            \itemsep1.5pt
            \item\label{algB:a} Setze $\DeltaMax = 0$.
            \item\label{algB:b} Setze $i = 1$.
            \item\label{algB:c} Solange $(i < n)$, wiederhole:\smallskip\\
                    \quad Setze \(test = \sfrac{(p_{i + 1} - p_i)}{p_i}\).\smallskip\\
                    \quad Wenn \(test > \DeltaMax\):~~Setze \(\DeltaMax = test\).\smallskip\\
                    \quad Erhöhe \(i\) um \(1\).
            \item\label{algB:d} Ergebnis ist $100 \cdot \DeltaMax$.
        \end{enumerate}
    }
    \begin{enumerate}[a)]
        \setcounter{enumi}{4}
        \item \pause Aufwandsanalyse: \begin{enumerate}[1.]
                  \item \pause Tabellarisch, siehe Vorlesung, oder:
                  \item \pause \say{Textbasiert}. Wir betrachten Stückweise: \begin{itemize}[<+(1)->]
                      \item Die ersten Zuweisungen \ref{algB:a} \& \ref{algB:b} sind  exakt zwei Elementaroperation.
                      \item Die äußere Schleife in \ref{algB:c} wird genau \(n - 1\) mal durchlaufen. \begin{description}[\textbullet\space]
                        \item[\textbullet] Erste Schleifenanweisung bedarf dreier Elementaroperationen: Subtraktion, Division \& Zuweisung.
                        \item[\textbullet] Zweite Schleifenanweisung ist ein Vergleich und maximal eine Zuweisung.
                        \item[\textbullet] Dritte Schleifenanweisung ist eine Elementaroperation (je nach Definition auch zwei).
                      \end{description}
                      \item Das Endergebnis ist eine Multiplikation (und je nach Definition eine Zuweisung).
                  \end{itemize}
                  \item\pause Damit erhalten wir: \(2 + (n - 1) \cdot ( 3+ 2 + 1) + 2 + 2 = 6n - 2\)\pause{} (also \(\mathcal{O}(n)\)).
              \end{enumerate}
    \end{enumerate}
\end{frame}

\iffull
\section{Aussicht: IEEE 754}
\begin{frame}{Aussicht: Das IEEE 754 Format}
    \begin{itemize}[<+(1)->]
        \item Erlaubt es Gleitkommazahl in Bits darzustellen
        \item Vorteil zu Festkommaformat: flexibler
        \item Beispiel mit wissenschaftlicher Notation: $ + 1.42 \cdot 10^{12}$\smallskip
        \item Grundlegend gilt für eine Zahl $a = s \cdot m \cdot b^e$, wobei: \begin{itemize}
                  \item[$s$] Signum, Vorzeichen, Plus oder Minus
                  \item[$m$] Mantisse an $\abs{m}$ Stellen (wie $1.42$), später genauer
                  \item[$b$] Basis, normiert, für uns: $10$, im IEEE 754: $2$
                  \item[$e$] Exponent an $\abs{e}$ Stellen.
              \end{itemize}
        \item Da Exponent immer positiv, existiert noch ein Bias $B$, ebenfalls definiert (meist auf die \say{Hälfte} das maximal möglichen Exponenten)
        \item Die Mantisse wird für $b = 2$ auf $1.$ (\say{Eins komma}) normiert, es werden nur die Stellen nach dem Komma gespeichert.
    \end{itemize}
\end{frame}

\begin{frame}{Das IEEE 754 Format - Ein Beispiel}
    \begin{itemize}[<+(1)->]
        \item Konvertiere $7.25$ ins IEEE 754 Format, wobei $\abs{m} = 5$, $\abs{e} = 2$. Der Bias sei $1$:
        \begin{enumerate}[1.]
                  \item positiv $\Rightarrow$ $S = 0$ [$s = (-1)^S$]
                  \item konvertiere die Zahl in das Binärsystem (nach dem Komma gilt $2^{-1}2^{-2}2^{-3}\ldots$):\pause{} \(7.25_{(10)} = 111.01\)
                  \item Normalisiere Mantisse: $111.01 =\pause 1.1101 \cdot 2^2 \Rightarrow M = 1101$
                  \item Berechne Exponent: $e = 2 \overset{+B = 1}{\Rightarrow}\pause E = 3_{(10)} = 11_{(2)}$
                  \item Überraschung, es geht alles auf \Smiley (wenn nicht, muss dies kenntlich gemacht werden, es existieren Rundungsverfahren!)
                  \item Gebe Ergebnis an ($S = 0$, $E = 11$, $M = 1101$). \pause{}
                        Wir erhalten: \[\underbrace{0}_{S} \underbrace{11}_{E} \underbrace{11010}_{M} = 01111010\vspace*{-0.25cm}\]
                        \pause{} Die Mantisse wird logisch (rechts!!!) mit Nullen aufgefüllt ($8$ Bit, jay \Smiley).
                  \item Hinweis: Zahlen wie die '0', Unendlich, NaN (Not a Number) besitzen besondere Darstellungen.
              \end{enumerate}
        \item Beachte \emph{Verlust}, Wenn Mantisse oder Exponent zu groß/klein!
    \end{itemize}
\end{frame}

\begin{frame}{Das IEEE 754 Format - Die Formate}
    \pause Hier die wichtigsten IEEE 754 Formate (dies so wirklich gibt) zum Sehen und
    Vergessen:\pause{}
    \begin{table}[H]
        \begin{mtabular}{l*4{c}}
            Typ & Größe & Exponent & Mantisse & Bias \\
            single & 32 bit & 8 bit & 23 bit & 127 \\
            double & 64 bit & 11 bit & 52 bit & 1023 \\
            \multicolumn{5}{c}{\vdots} \\
            single ext, min & 43 bit & 11 bit & 31 bit & 1023 \\
            double ext. min & 79 bit & 15 bit & 63 bit & 16383 \\
        \end{mtabular}
    \end{table}
\end{frame}

\begin{frame}{Das IEEE 754 Format - Ein Beispiel, reloaded}
    \begin{itemize}[<+(1)->]
        \item \strut Konvertiere \(01111010\) aus dem IEEE 754 Format ins Fließkommaformat, wobei \(\abs{m} = 5\) und \(\abs{e} = 2\). Der Bias sei $1$:
        \begin{enumerate}[1.]
                  \item Erstes Bit $0$ $\Rightarrow$ positiv
                  \item Betrachte nächste beiden Bits (da $\abs{e} = 2$): $11_{(2)} \widehat{=} 3_{(10)}$\pause{} Bias abziehen ergibt: $3 - 1 = 2$.
                  \item Isoliere Mantisse: $11010$ (da $\abs{m} = 5$), füge \say{Eins komma} an:\pause{} $1.11010$\pause{} multipliziere über Basis ($=2$) mit Exponent:\pause{} $1.1101 \cdot 2^2 \pause{}= 111.01$.
                  \item Rechne um: $111.01_{(2)} \pause{}\;\widehat{=}\; 7 + 0 \cdot 2^{-1} + 1 \cdot 2^{-2} = 7.25_{(10)}$
                  \item Füge Vorzeichen an (hier positiv):\pause{} $7.25$, ferddisch! (und schdimmd, \Laughey)
              \end{enumerate}
    \end{itemize}
\end{frame}

\begin{frame}{Das IEEE 754 Format - Rundungsproblematik}
    \begin{itemize}[<+(1)->]
        \item Da mit $2^{-1}2^{-2}2^{-3}\ldots$ nicht jede (dezimale) Fließkommazahl exakt darstellbar.
        \item Beispiel: \pause
              \ijava{RoundingProblem.java}
        \item Erwartung: $0.1$ und $0.1$
    \end{itemize}
\end{frame}

\begin{frame}{Das IEEE 754 Format - Rundungsproblematik}
    \begin{itemize}
        \rExecute{javac RoundingProblem.java}%
        \item \textbf{Leider nein}:\Rbash[:\pause]{java RoundingProblem}{java RoundingProblem}
        \item\pause{}Warum?\pause{}\newline{}$0.1_{(10)} \pause{}\,\widehat{=}\, 0.000110011001100110011\ldots_{(2)} \pause\neq 0.1_{(10)}$ da Mantisse begrenzt.
    \end{itemize}
\end{frame}

\section{Bonus: Weitere Algorithmusanalyse}


\begin{frame}[c]{Algorithmenbau}
    \pause\centering \parbox{.65\linewidth}{\footnotesize Gegeben sei folgende Situation: Es soll ein kleine Party veranstaltet werden, zu der Sie möglichst viele Ihrer Bekannten
    einladen wollen. Jedoch gibt es hier verschiedene Bedingungen, die zu beachten sind: zum einen kommen Einige nur
    dann, wenn auch eine andere Person eingeladen wird. Andere wiederum können bestimmte Personen nicht leiden und
    kommen nicht, wenn eine bestimmte Person auch eine Einladung erhält. Nehmen Sie an, dass sie bereits eine Liste
    dieser Bedingungen in Form von Aussagen \say{X kommt nur, wenn Y kommt} oder \say{X kommt nicht, wenn Y kommt} aufgeschrieben haben.\pause\bigskip\par \footnotesize
        Entwickeln Sie einen Algorithmus, der Ihnen aus der Liste Ihrer Bekannten und der Liste an Bedingungen eine Einlandungsliste berechnet, die möglichst viele Personen einlädt.
    }\qquad\parbox{.25\linewidth}{%
        \scriptsize
        \onslide<4->{\begin{enumerate}[a)]
            \item Problemspezifikation
            \item Problemabstraktion
            \item Algorithmenentwurf
            \item Korrektheitsnachweis
            \item Aufwandsanalyse
        \end{enumerate}}
    }
\end{frame}

\begin{frame}[c]{Algorithmenbau}
    \pause\centering \parbox{.65\linewidth}{\only<4->{\color{gray}}\footnotesize Gegeben sei folgende Situation: Es soll ein kleine Party veranstaltet werden, zu der Sie möglichst viele Ihrer Bekannten
    einladen wollen. Jedoch gibt es hier verschiedene Bedingungen, die zu beachten sind: zum einen kommen Einige nur
    dann, wenn auch eine andere Person eingeladen wird. Andere wiederum können bestimmte Personen nicht leiden und
    kommen nicht, wenn eine bestimmte Person auch eine Einladung erhält. Nehmen Sie an, dass sie bereits eine Liste
    dieser Bedingungen in Form von Aussagen \say{X kommt nur, wenn Y kommt} oder \say{X kommt nicht, wenn Y kommt} aufgeschrieben haben.\pause\bigskip\par \footnotesize
        \only<4->{\color{gray}}Entwickeln Sie einen Algorithmus, der Ihnen aus der {\only<4->{\bfseries\color{paletteA}}Liste Ihrer Bekannten} {\only<4->{\color{gray}}und} der {\only<4->{\bfseries\color{paletteA}}Liste an Bedingungen} eine {\only<4->{\bfseries\color{paletteC}}Einlandungsliste berechnet}, die möglichst viele Personen einlädt.
    }\only<5->{\qquad\parbox{.25\linewidth}{%
        \centering\footnotesize \onslide<6->{\paletteA{Bekanntenliste}\\{\color{gray}\((b_1,\;\ldots,\;b_n)\)}\par
        \onslide<7->{\paletteA{Bedingungsliste}\\{\color{gray}\((c_1,\;\ldots,\;c_k)\)}\medskip\par}
        \onslide<8->{\faCaretDown\medskip\par \paletteC{Einladeliste}\\{\color{gray}\((e_1, \ldots,\;e_j)\)}}}
    }}
\end{frame}

{\def\g#1{{\only<2->{\color{black}}#1}}\def\k#1{{\only<3->{\bfseries\color{paletteA}}#1}}
\begin{frame}[c]{Algorithmenbau}
    \centering \parbox{.65\linewidth}{\color{gray}\footnotesize Gegeben sei folgende Situation: Es soll ein \g{kleine Party} veranstaltet werden, zu der Sie möglichst viele Ihrer \g{\k{Bekannten}}
    einladen wollen. Jedoch gibt es hier verschiedene \g{\k{Bedingungen}}, die zu beachten sind: zum einen \g{kommen} Einige nur
    dann, wenn auch eine andere Person eingeladen wird. Andere wiederum können bestimmte \g{Personen} nicht leiden und
    kommen nicht, wenn eine bestimmte Person auch eine \g{\k{Einladung}} erhält. Nehmen Sie an, dass sie bereits eine Liste
    dieser Bedingungen in Form von \g{Aussagen} \say{X kommt nur, wenn Y kommt} oder \say{X kommt nicht, wenn Y kommt} aufgeschrieben haben.\bigskip\par \footnotesize
    Entwickeln Sie einen Algorithmus, der Ihnen aus der \g{Liste} Ihrer Bekannten und der Liste an Bedingungen eine Einlandungsliste berechnet, die \g{\k{möglichst viele}} Personen einlädt.
    }\qquad\parbox{.25\linewidth}{%
        \centering\footnotesize Begriffe:\medskip\\
        \onslide<4->{Bekannten\textcolor{gray}{liste}}\\
        \onslide<5->{Bedingungs\textcolor{gray}{liste}}\\
        \onslide<6->{Einladungs\textcolor{gray}{liste}}\\
        \onslide<7->{möglichst viele}\\
    }
\end{frame}
}

\begin{frame}{Algorithmenbau a) \& b)}
    \begin{enumerate}[a)]
        \item\pause Problemspezifikation: \begin{itemize}[<+(1)->]
            \item \emph{Bekanntenliste:} eine Menge von Personen \info{z.B. als Zeichenketten der Namen}.
            \item \emph{Einladungsliste:} eine Teilmenge der Menge an bekannten Personen.
            \item \emph{Möglichst viele:} maximale Größe der Menge eingeladener Bekannter.
            \item \emph{Bedingung:} eine eindeutig überprüfbare Aussage \info{hier über die Einladungsliste}.
            \item \emph{Bedingungsliste:} eine Menge an Bedingungen.
        \end{itemize}
        \pause\par
        {\scriptsize\info{Warum Mengen und nicht mehr Listen? Für jede Liste können wir Argumentieren, dass Dopplungen unsinnig sind: Doppelte Einladungen sind ebenso sinnfrei wie doppelte Bedingungen. Damit lockern wir so das Problem ein wenig auf und vereinfachen uns die Lösung.}\par}
        \item\pause Problemabstraktion: \begin{itemize}[<+(1)->]
            \item \emph{Gegeben:} (\(n, k \in \N\)) \begin{enumerate}[1.]
                \item Eine endliche Menge von Bekannten \(B = \{b_1, \ldots, b_n\}\) \info{hoffentlich mit \(B \neq \emptyset\) \Laughey}.
                \item Eine endliche Menge an Bedingungen \(C = \{c_1, \ldots, c_k\}\).
            \end{enumerate}
            \item \emph{Gesucht:} Eine Menge einzuladender Bekannten \(E \subseteq B\), die alle Bedingungen \(c \in C\) erfüllt, mit maximalem \(\abs{E}\).
        \end{itemize}
    \end{enumerate}
\end{frame}

\begin{frame}{Algorithmenbau c) \& d)}
    \begin{enumerate}[a)]
        \setcounter{enumi}{2}
        \item Algorithmenentwurf: \info{Wir probieren einfach jede Konstellation}% Note => SAT-Solving
        \null\quad\parallelcontent[c]{\setbeamercolor{itemize/enumerate subbody}{fg=black}%
        \begin{enumerate}[(1)]
            \item\pause Setze $Max = \emptyset$.
            \item\label{alg:st}\pause Für alle $E \in \P(B)$ (alternativ: \(E \subseteq B\)):\\
                \pause \quad Wenn ($\abs{E} > \abs{Max}$):\\
                \pause \qquad Für alle \(c \in C\):\\
                \pause \quad\qquad Wenn (\(E\) erfüllt nicht \(c\)): Weiter über (\ref{alg:st}).\\
                \pause \qquad Setze \(Max = E\).
            \item\pause Ergebnis ist $Max$.
        \end{enumerate}
        }{\pause\footnotesize\color{gray}%
            \say{E erfüllt nicht \(c\)} können wir unterschiedlichst testen. Ganz naiv genügt hier ein Test abhängig von der Form: \begin{enumerate}[(1)]
                \item \say{X nur, wenn Y}:\\ \(c\) scheitert, wenn \(X \in E\) aber \(Y \not\in E\).
                \item \say{X nicht, wenn Y}:\\ \(c\) scheitert, wenn \(X \in E\) und \(Y \in E\).
            \end{enumerate}
        }\medskip% Wir können Optimieren: einfach von oben herab
        \item\pause Korrektheitsnachweis, Verifikation:\begin{enumerate}[1.]
                  \item \pause Vollständige Induktion, oder:
                  \item \pause\say{Textbasiert}.\pause{} \(B\) und \(C\) sind endliche und konstante Mengen (damit ist auch \(\P(B)\) endlich und konstant).\pause{} Daraus werden auch die Schleifen nur endlich oft durchlaufen. Der Algorithmus \textit{terminiert} somit.\smallskip

                  \pause Zudem ist er \textit{partiell korrekt}, in jedem Durchlauf wird \(Max\) genau dann geändert, wenn \(E\) größer ist \emph{und} alle \(c \in C\) erfüllt.\pause{} Da alle möglichen Konstellationen \(E \subseteq B\) geprüft werden, findet sich so das \(E\) mit maximaler Kardinalität.
              \end{enumerate}
    \end{enumerate}
\end{frame}

\begin{frame}{Algorithmenbau e)}
    {}{\tiny
        \parallelcontent{\setbeamercolor{itemize/enumerate subbody}{fg=black}%
        \vspace*{-\baselineskip}\begin{enumerate}[(1)]
            \item\pause\label{algC:a} Setze $Max = \emptyset$.
            \item\label{algC:b} Für alle $E \in \P(B)$ (alternativ: \(E \subseteq B\)):\\
                \quad Wenn ($\abs{E} > \abs{Max}$):\\
                \qquad Für alle \(c \in C\):\\
                \quad\qquad Wenn (\(E\) erfüllt nicht \(c\)): Weiter über (\ref{alg:st}).\\
                \qquad Setze \(Max = E\).
            \item\label{algC:c} Ergebnis ist $Max$.
        \end{enumerate}
    }{%
        \pause\medskip\par\footnotesize\color{gray}%
        Aus der Mathematik ist bekannt, dass für eine endliche Menge \(\abs{\P(X)} = 2^{\abs{X}}\).
    }}\bigskip
    \begin{enumerate}[a)]
        \setcounter{enumi}{4}
        \item \pause Aufwandsanalyse: \begin{enumerate}[1.]
                  \item \pause Tabellarisch, siehe Vorlesung, oder:
                  \item \pause \say{Textbasiert}. Wir betrachten Stückweise: \begin{itemize}[<+(1)->]
                      \item Die erste Zuweisung \ref{algC:a} ist exakt eine Elementaroperation.
                      \item Die äußere Schleife in \ref{algC:b} wird \(2^{\abs{B}}\) mal durchlaufen.
                      \item Die erste Bedingung bedarf 3 Elementaroperationen: 2 Kardinalitätsberechnungen, 1 Vergleich.
                      \item Die innere Schleife wird (bis zu) \(\abs{C}\) mal durchlaufen.
                      \item Die innere Schleife enthält eine Elementaroperation.
                      \item Die äußere Schleife enthält weiter noch eine Elementaroperation (\(Max = E\)).
                  \end{itemize}
                  \item\pause Damit erhalten wir: \(1 + 2^{\abs{B}} \cdot (3 + \abs{C} \cdot (1) + 1) = 1 + 2^{\abs{B}} \cdot (4 + \abs{C})\)
              \end{enumerate}
    \end{enumerate}
\end{frame}

\fi

\section{End Pinguuu}
\begin{frame}[c]
    \hbox{}\vfill
    \centering\onslide<2->{\begin{tikzpicture}[scale=1.9]
            \pingu[name = herbertine, heart=purple!15!pingu@white, left wing hug, right wing wave,eyes shiny]
            \node[above, inner sep=1.5pt,scale=4, pingu@red!70!white] at(herbertine-wing-right-tip) {\faHeart};
            % \HerbertMarkers
        \end{tikzpicture}\begin{tikzpicture}[scale=1.9]
            \pingu[name = herbert, heart=teal!15!pingu@white, wings shock, eyes shock,pumpkin-hat,pumpkin-hat outline=gray!70!pingu@bronze]
            % \HerbertMarkers
        \end{tikzpicture}}
\end{frame}
\iffull\end{document}\fi

\InputIfFileExists{../global.src}\relax\relax

\iffull
\def\smlll{\texorpdfstring{\def\RSsmallest{2pt}\smaller[2]}{}}
\title[Zehntes Tutorium -- Übungsblatt 10]{Ich sortiere das. \textit{Ich}, als Sortierer.\\\small Ordnung 10}
\date{\sffamily KW 4}

\usepackage[glows]{tikzpingus}
\usetikzlibrary{decorations.text,matrix}
\hypersetup{colorlinks=false}

\begin{document}
\Titlepage{10}
\fi


\section{Präsenzaufgabe}
\iffull
\begin{frame}[c]{Präsenzaufgabe}
    \begin{aufgabe}{Bitte anstellen\ldots}
        \only<-5|handout:0>{\pause{}\small In dieser Aufgabe sollen Sie ihre eigene dynamische Datenstruktur implementieren. Hierbei solle es sich um eine Queue handeln, welche nach dem \textit{FIFO}-Prinzip (\say{first in first out}) arbeitet und Integer Werte speichert.\pause{} Intern benötigen Sie hier zusätzlich zur Klasse \bjava{Queue} eine innere Klasse um die Elemente repräsentieren zu können.

        \pause Implementieren Sie einen Konstruktor für die Klasse \textit{Queue}, welcher die internen Variablen entsprechend initialisiert und zusätzlich die folgenden Methoden: \only<4|handout:0>{\begin{enumerate}
            \item \bjava{public void enqueue(int value)} --- soll ein neues Element mit dem Wert \bjava{value} am Ende der Queue einreihen.
            \item \bjava{public boolean dequeue()} --- soll das vorderste Element aus der \bjava{Queue} entfernen und zurückliefern ob das Entfernen möglich war.
            \item \bjava{public int getLength()} --- soll die Anzahl der eingereihten Elemente zurück geben.
            \item \bjava{public String toString()} --- soll die eingereihten Werte in einem String formatieren und diesen Zurückgeben.
        \end{enumerate}}

        \pause \textit{Bonusfrage}: Angenommen, Sie wollten die Methode \bjava{public :lan:?:ran: head()} implementieren, die den Wert des ersten Elements zurückgibt. Welchen Rückgabetyp würden sie wählen? Warum?}%
        \only<6->{\vspace*{-1.25\baselineskip}\par%
            \small Implementieren Sie eine (Linked-)\bjava{Queue} Klasse, die nach dem \textit{FIFO}-Prinzip (\say{first in first out}) Integer-Werte verwaltet. Schreiben Sie dafür eine interne Klasse um die Elemente zu repräsentieren. Implementieren Sie weiter:\vspace*{-.33\baselineskip} \onslide<7->{\begin{itemize}
                \itemsep0.5pt
                \item Den Konstruktor \bjava{public Queue()}, der alles \textit{explizit} zuweist.
                \item \bjava{public void enqueue(int value)} --- füge \bjava{value} am Ende an.
                \item \bjava{public boolean dequeue()} --- entferne das vorderste Element (\bjava{false} wenn leer).
                \item \bjava{public int getLength()} --- Anzahl der Elemente in der Queue.
                \item \bjava{public String toString()} --- die Inhalte als String formatiert.
            \end{itemize}}\vspace*{-.33\baselineskip}\par
            \onslide<8->{\textit{Bonus:} Welchen Rückgabetyp würden Sie für Die Methode \bjava{public :lan:?:ran: head()} wählen, die den Wert des ersten Elements zurückgibt. Warum? \hfill\textbf{Hilfsmethoden sind erlaubt \Smiley.}}%
        }%
        \onslide<1->
    \end{aufgabe}
    \only<5|handout:0>{
        \begin{tikzpicture}[overlay, remember picture]
            \node[above left=.5cm,text width=6cm,font=\footnotesize] at(current page.south east) {\begin{enumerate}
                \item \bjava{public void enqueue(int value)} --- füge am Ende an.
                \item \bjava{public boolean dequeue()} --- entferne vorderstes Element.
                \item \bjava{public int getLength()} --- Anzahl eingereihter Elemente.
                \item \bjava{public String toString()} --- Formatiert als String.
            \end{enumerate}};
        \end{tikzpicture}
    }
    \onslide<9->{\begin{tikzpicture}[overlay,remember picture]
        \fill[pingu@black,opacity=.985] (current page.south east) rectangle (current page.north west);
        \node[above right,scale=.35] at (current page.south west) {Sonntag 20:37 Uhr und ich darf das alles nochmal anpassen\ldots};
        \onslide<10->{\node at (current page.center) {\usebox\donotmesswithbeate};}
    \end{tikzpicture}}
\end{frame}
\fi

\newsavebox\QueueShort \savebox\QueueShort{\begin{minipage}{8cm}
\begin{plainjava}[aboveskip=1pt,belowskip=1pt]^^J
class Queue \{^^J
:ws::ws:class Element \{^^J
:ws::ws::ws::ws:public final int value;^^J
:ws::ws::ws::ws:public Element next = null;^^J
:ws::ws::ws::ws:public Element(int value) \{^^J
:ws::ws::ws::ws::ws::ws:this.value = value;^^J
:ws::ws:\}:ws:\}^^J
:ws::ws:// :ldots:^^J
\}^^J
\end{plainjava}
\vspace*{-.3\baselineskip}
\end{minipage}}
\newsavebox\queueboxtask
\begin{frame}[c,fragile]{Präsenzaufgabe}
    \savebox\queueboxtask{%
\begin{minipage}{\linewidth}
\begin{aufgabe}{Bitte anstellen\ldots\ die Zweite}
    \vspace*{-1.25\baselineskip}\par%
    \small Implementieren Sie eine (Linked-)\bjava{Queue} Klasse, die nach dem \textit{FIFO}-Prinzip (\say{first in first out}) Integer-Werte verwaltet. Implementieren Sie dafür:\vspace*{-.33\baselineskip} \begin{itemize}
            \itemsep0.5pt
            \item Den Konstruktor \bjava{public Queue()}, der alles \textit{explizit} zuweist.
            \item \bjava{public void enqueue(int value)} --- füge \bjava{value} am Ende an.
            \item \bjava{public boolean dequeue()} --- entferne das vorderste Element (\bjava{false} wenn leer).
            \item \bjava{public int getLength()} --- Anzahl der Elemente in der Queue.
            \item \bjava{public String toString()} --- die Inhalte als String formatiert.
        \end{itemize}\vspace*{-.33\baselineskip}\par
        \textit{Bonus:} Welchen Rückgabetyp würden Sie für Die Methode \bjava{public :lan:?:ran: head()} wählen, die den Wert des ersten Elements zurückgibt. Warum? \hfill\textbf{Hilfsmethoden sind erlaubt \Smiley.}%
\end{aufgabe}\end{minipage}}
    \only<-2|handout:0>{\usebox\queueboxtask}
    \onslide<2->{\begin{tikzpicture}[overlay,remember picture]
        \onslide<-3|handout:0>{\node[above left=-3.75cm] at (current page.south east) {\rotatebox{30}{\usebox\donotmesswithbeate}};}
        \onslide<4->{\node[above left=-5.95cm] at (current page.south east) {\rotatebox{30}{\usebox\donotmesswithbeate}};}
        \onslide<3->{\node[below=-9mm,xshift=-8mm] at(current page.north) {\rotatebox{5}{\usebox\queueboxtask}};}
        \onslide<5->{\node[above left,draw=gray,very thick,fill=white,rounded corners=2pt,scale=.75] at ([xshift=-1cm,yshift=.45cm]current page.south east) {\usebox\QueueShort};}
        \onslide<6->{\node[above,draw=gray,very thick,fill=white,rounded corners=2pt,text width=7.5cm] at([xshift=-2.66cm,yshift=1cm]current page.south) {\small Verwenden Sie für ihre Implementierung keine Arrays oder vorgefertigten dynamischen Datenstrukturen, sondern implementieren Sie die Queue selbst als (einfach) verkettete Liste.\par};}
    \end{tikzpicture}}
\end{frame}

\begin{frame}[fragile]{Präsenzaufgabe - Lösung\iffull\hfill\null\llap{\smash{\raisebox{-.3\baselineskip}{\scalebox{.7}{\itshape\footnotesize Or how to make \(\sim\)100 Lines Musterlösung verkraftbar}}}}\fi}
    \SetupLstHl
    \begin{itemize}[<+(1)->]
        \item Wir beginnen mit der Grundstruktur: \only<2->{\textattachfile{\curpath Queue.java}{Queue.java}}
    \end{itemize}
\begin{plainjava}
!*\onslide<3->*!public class Queue {
!*\onslide<4->*!    class Element {
!*\onslide<5->*!        private final int value;
!*\onslide<6->*!        private Element next;
!*\onslide<7->*!        // :ldots:
!*\onslide<4->*!    }
!*\onslide<3->*!
!*\onslide<8->*!    private Element first;
!*\onslide<8->*!    private Element last;
!*\onslide<8->*!    private int length;
!*\onslide<9->*!    // :ldots:
!*\onslide<3->*!}
\end{plainjava}
\end{frame}

\MakeThePinguExplainIt[text width=6.65cm]{cap=!hide,hat,glasses=!hide,eyes wink,cup=!hide,heart=shadeA,bronze medal,right item angle=5}{Oder kompakter:
\begin{plainjava}^^J
class Element \{^^J
:ws::ws:public final int value;^^J
:ws::ws:public Element next = null;^^J
:ws::ws:public Element(int value) \{^^J
:ws::ws::ws::ws:this.value = value;^^J
:ws::ws:\}^^J
\}^^J
\end{plainjava}}
\begin{frame}[c,fragile]{Präsenzaufgabe - Lösung\hfill Element - Alternative}
\lstfs{10}\lstset{lineskip=2.25pt}\SetupLstHl
\begin{plainjava}
!*\onslide<2->*!|ihl|public class Queue {|ihl|
!*\onslide<2->*!    class Element {
!*\onslide<2->*!        private final int value;
!*\onslide<2->*!        private Element next;
!*\onslide<3->*!        public Element(int value) {
!*\onslide<4->*!            this.value = value;
!*\onslide<4->*!            this.next = null;
!*\onslide<3->*!        }
!*\onslide<3->*!
!*\onslide<5->*!        public void setNext(Element next) { this.next = next; }
!*\onslide<6->*!        public Element getNextElement() { return this.next; }
!*\onslide<7->*!        public int getValue() { return this.value; }
!*\onslide<2->*!    }
!*\onslide<2->*!    // :ldots:
!*\onslide<2->*!|ihl|}|ihl|
\end{plainjava}
\begin{tikzpicture}[overlay,remember picture]
\onslide<8->{\node[left=-7mm,scale=.8] at(current page.-1) {\usebox\pinguexplainbox};}
\end{tikzpicture}
\end{frame}

\begin{frame}[c,fragile]{Präsenzaufgabe - Lösung\hfill Queue - Konstruktor}
\SetupLstHl
\begin{plainjava}
!*\onslide<2->*!|ihl|public class Queue {|ihl|
!*\onslide<2->*!    |ihl|class Element { :ldots: }|ihl|
!*\onslide<2->*!    |ihl|private Element first, last;|ihl|
!*\onslide<2->*!    |ihl|private int length;|ihl|

!*\onslide<3->*!    public Queue() {
!*\onslide<4->*!        this.first = null;
!*\onslide<4->*!        this.last = null;
!*\onslide<4->*!        this.length = 0;
!*\onslide<3->*!    }
!*\onslide<2->*!|ihl|}|ihl|
\end{plainjava}
\end{frame}

\begin{frame}[c,fragile]{Präsenzaufgabe - Lösung\hfill Queue - Enqueue}
\SetupLstHl\lstfs{10}
\begin{plainjava}
!*\onslide<2->*!|ihl|public class Queue {|ihl|
!*\onslide<2->*!    |ihl|class Element {|ihl|
!*\onslide<2->*!    |ihl|    public final int value;  public Element next = null;|ihl|
!*\onslide<2->*!    |ihl|    public Element(int value) { this.value = value; }|ihl|
!*\onslide<2->*!    |ihl|}|ihl|
!*\onslide<2->*!    |ihl|private Element first, last;   private int length;|ihl|
!*\onslide<3->*!    public void enqueue(int value) {
!*\onslide<4->*!        Element newElement = new Element(value);
!*\onslide<5->*!        if(length == 0) { // Sonderfall: Erstes
!*\onslide<6->*!            this.first = this.last = newElement;
!*\onslide<5->*!        } else {
!*\onslide<7->*!            this.last.next = newElement;
!*\onslide<7->*!            this.last = newElement; // Verschieben
!*\onslide<5->*!        }
!*\onslide<8->*!        length++;
!*\onslide<3->*!    }
!*\onslide<2->*!|ihl|}|ihl|
\end{plainjava}
\end{frame}

\begin{frame}[c,fragile]{Präsenzaufgabe - Lösung\hfill Queue - Dequeue}
\SetupLstHl\lstfs{10}
\begin{plainjava}
!*\onslide<2->*!|ihl|public class Queue {|ihl|
!*\onslide<2->*!    |ihl|class Element {|ihl|
!*\onslide<2->*!    |ihl|    public final int value;  public Element next = null;|ihl|
!*\onslide<2->*!    |ihl|    public Element(int value) { this.value = value; }|ihl|
!*\onslide<2->*!    |ihl|}|ihl|
!*\onslide<2->*!    |ihl|private Element first, last;   private int length;|ihl|
!*\onslide<3->*!    public boolean dequeue() {
!*\onslide<4->*!        if(length > 0) {
!*\onslide<5->*!            this.first = this.first.next; // Verschieben
!*\onslide<6->*!            length--;
!*\onslide<7->*!            if(length == 0)
!*\onslide<7->*!                this.last = null;
!*\onslide<8->*!            return true;
!*\onslide<4->*!        }
!*\onslide<9->*!        return false;
!*\onslide<3->*!    }
!*\onslide<2->*!|ihl|}|ihl|
\end{plainjava}
\end{frame}

\begin{frame}[c,fragile]{Präsenzaufgabe - Lösung\hfill Queue - getLength \& toString}
\SetupLstHl\lstfs{10}
\begin{plainjava}
!*\onslide<2->*!|ihl|public class Queue {|ihl|
!*\onslide<2->*!    |ihl|class Element {|ihl|
!*\onslide<2->*!    |ihl|    public final int value;  public Element next = null;|ihl|
!*\onslide<2->*!    |ihl|    public Element(int value) { this.value = value; }|ihl|
!*\onslide<2->*!    |ihl|}|ihl|
!*\onslide<2->*!    |ihl|private Element first, last;   private int length;|ihl|
!*\onslide<3->*!    public int getLength() { return this.length; }
!*\onslide<4->*!    public String toString() {
!*\onslide<5->*!        if(length <= 0) return "[]";
!*\onslide<4->*!
!*\onslide<6->*!        String s = "[" + this.first.value;
!*\onslide<7->*!        for (Element cur = this.first.next; cur != null; cur = cur.next) {
!*\onslide<8->*!            s += ", " + cur.value;
!*\onslide<7->*!        }
!*\onslide<9->*!        return s + "]";
!*\onslide<4->*!    }
!*\onslide<2->*!|ihl|}|ihl|
\end{plainjava}
\end{frame}

\begin{frame}[c,fragile]{Präsenzaufgabe - Lösung\hfill Bonus}
\begin{itemize}[<+(1)->]
    \itemsep9pt
    \item Doch was machen wir mit \bjava{public :lan:?:ran: head()}?
    \item Idee: \bjava{int}. \begin{itemize}
        \item Problem: Was ist, wenn die Queue kein Element hat?
        \item Möglichkeit: Wir können eine Exception Werfen!
    \end{itemize}
    \item Idee: \bjava{Integer}. \begin{itemize}
        \item Warum? Es erlaubt \bjava{null} als Rückgabe.
        \item Problem: Wir sollten kein \bjava{null} zurückgeben.
        \item Abhilfe: Für solche Fälle gibt es \bjava{Optional<T>}.
    \end{itemize}
    \item Idee: Ein eigener ternärer/optionaler Wert (\only<10->{\textattachfile{\curpath MayQueueValue.java}{MayQueueValue.java}}).\lstfs{9}
\begin{plainjava}
public class MayQueueValue {
    private final boolean stored;  private final int value;
    // :ldots:
}
    \end{plainjava}
    \begin{itemize}
        \item Problem: Wirkt ein wenig Overkill für etwas, das eigentlich illegal ist!~\(\longrightarrow\) Exception!
    \end{itemize}
\end{itemize}
\end{frame}

\section{Übungsblatt 10}
\subsection{Aufgabe 1}
\iffull
\begin{frame}[c,fragile]{Übungsblatt 10 - Aufgabe 1}
    \pause Sortieren Sie die Zahlenfolge \(8, 6, 3, 5, 1, 2, 7, 4\) grafisch mit den folgenden Sortieralgorithmen:
    \begin{itemize}[<+(1)->]
        \item Mergesort
        \item Quicksort --- Nutzen Sie jeweils das erste Element als Pivotelement.
        \item Insertionsort
    \end{itemize}
    \bigskip\par\pause

    \resizebox{2cm}!{\tikz[align-half-base]\pingu[eyes wink,left wing wave, sign post left={Note!}];}\hfill\parbox{\linewidth-2cm}{%
    Ich nehmen für die Folien an, dass wir immer \textit{aufsteigend} sortieren. Alle andere Sortierweisen (wie \textit{absteigend}) gehen analog.}
\end{frame}
\fi

\begin{frame}[c]{Übungsblatt 10 - Aufgabe 1\hfill Mergesort}
    \begin{center}
\resizebox{.975\linewidth}{!}{%
\begin{tikzpicture}[yscale=0.9]
    \onslide<1->{\node (a) at(0,0) {[\(8, 6, 3, 5, 1, 2, 7, 4\)]};}
    \foreach[count=\i] \l/\x in {{8,6,3,5}/-4,{1,2,7,4}/4}{
        \onslide<+(1)->{
            \node (b\i) at(\x,-1) {[\(\l\)]};
            \draw[-Kite] (a) -- (b\i);
        }
    }

    \foreach[count=\i] \l/\x/\p in {{8,6}/-6/1,{3,5}/-2/1,{1,2}/2/2,{7,4}/6/2}{
        \onslide<+(1)->{
            \node (c\i) at(\x,-2) {[\(\l\)]};
            \draw[-Kite] (b\p) -- (c\i);
        }
    }

    \foreach[count=\i] \l/\x/\p in {{8}/-7/1,{6}/-5/1,{3}/-3/2,{5}/-1/2,%
        {1}/1/3,{2}/3/3,{7}/5/4,{4}/7/4}{
        \onslide<+(1)->{
            \node (d\i) at(\x,-3) {[\(\l\)]};
            \draw[-Kite] (c\p) -- (d\i);
        }
    }

    \onslide<+(1)->{
    \foreach[count=\i] \l/\x/\p in {{8}/-7/1,{6}/-5/2,{3}/-3/3,{5}/-1/4,%
        {1}/1/5,{2}/3/6,{7}/5/7,{4}/7/8}{
            \node (e\i) at(\x,-4) {[\(\l\)]};
            \draw[densely dotted,-Kite,gray] (d\p) -- (e\i);
        }
    }

    \foreach[count=\i] \l/\x/\p/\pt in {{6,8}/-6/1/2,{3,5}/-2/3/4,{1,2}/2/5/6,{4,7}/6/7/8}{
        \onslide<+(1)->{
            \node (f\i) at(\x,-5) {[\(\l\)]};
            \ifnum\p=\pt\draw[densely dotted,gray,-Kite] (e\p) -- (f\i);\else \draw[-Kite] (e\p) -- (f\i);\draw[-Kite] (e\pt) -- (f\i); \fi
        }
    }

    \foreach[count=\i] \l/\x/\p/\pt in {{3,5,6,8}/-4/1/2,{1,2,4,7}/4/3/4}{
        \onslide<+(1)->{
            \node (g\i) at(\x,-6) {[\(\l\)]};
            \ifnum\p=\pt \draw[densely dotted,gray,-Kite] (f\p) -- (g\i);\else \draw[-Kite] (f\p) -- (g\i);\draw[-Kite] (f\pt) -- (g\i); \fi
        }
    }

    \foreach[count=\i] \l/\x/\p/\pt in {{1,2,3,4,5,6,7,8}/0/1/2}{
        \onslide<+(1)->{
            \node (h\i) at(\x,-7) {[\(\l\)]};
            \draw[-Kite] (g\p) -- (h\i); \draw[-Kite] (g\pt) -- (h\i);
        }
    }
    \onslide<+(1)->{\draw[densely dashed] (-8,-3.5) -- ++(17.25,0) node[above left] {\strut Divide} node[below left] {\strut Merge};}
\end{tikzpicture}
}
    \end{center}
\end{frame}

\begin{frame}[c]{Übungsblatt 10 - Aufgabe 1\hfill Quicksort}
    \begin{itemize}[<+(1)->]
        \itemsep12.5pt
        \item Quicksort kann unterschiedlich angewendet werden.
        \item Wir betrachten \iffull drei\else zwei\fi\ verschiedene Varianten: \begin{itemize}
            \iffull \item Eine einfache mit Listen (die direkte Umsetzung, genügt nicht).\fi
            \item Die Version der Vorlesung (Pivot ans Ende, Marker)
            \item Eine etwas schnellere Version (nur mit Markern).
        \end{itemize}
        \item \iffull Alle\else Beide\fi\ folgen dem dem bekannten Schema: \begin{itemize}
            \item Wenn Liste einelementig oder leer: fertig.
            \item Wähle Pivotelement und teile Liste in \say{kleiner} und \say{größergleich}.
            \item Wiederhole rekursiv für beide Teile.
        \end{itemize}
    \end{itemize}
\end{frame}


\iffull
\begin{frame}[c]{Übungsblatt 10 - Aufgabe 1\hfill Quicksort -- Listenversion}
    \begin{center}
\resizebox{.65\linewidth}{!}{%
\begin{tikzpicture}[yscale=0.9]
    \onslide<1->{\node (a) at(0,0) {[\(\pivotelem{8}, 6, 3, 5, 1, 2, 7, 4\)]};}
    \foreach[count=\i] \l/\x in {{\pivotelem{6}, 3, 5, 1, 2, 7, 4}/-4,{}/4}{
        \onslide<+(1)->{
            \node (b\i) at(\x,-1) {[\(\l\)]};
            \draw[-Kite] (a) -- (b\i);
        }
    }

    \foreach[count=\i] \l/\x/\p in {{\pivotelem3,5,1,2,4}/-6/1,{\pivotelem7}/-2/1}{
        \onslide<+(1)->{
            \node (c\i) at(\x,-2) {[\(\l\)]};
            \draw[-Kite] (b\p) -- (c\i);
        }
    }

    \foreach[count=\i] \l/\x/\p in {{\pivotelem1,2}/-7/1,{\pivotelem5,4}/-5/1}{
        \onslide<+(1)->{
            \node (d\i) at(\x,-3) {[\(\l\)]};
            \draw[-Kite] (c\p) -- (d\i);
        }
    }

    \foreach[count=\i] \l/\x/\p/\c in {{\pivotelem2}/-6.5/d1/{},{\pivotelem4}/-5.5/d2/{}}{
        \onslide<+(1)->{
            \node (e\i) at(\x,-4) {[\(\l\)]};
            \expandafter\draw\expandafter[\c,-Kite] (\p) -- (e\i);
        }
    }

    \foreach[count=\i] \l/\x/\p/\px/\pt in {{\pivotelem1,2}/-7/1/d1/1,{4,\pivotelem5}/-5/2/d2/2}{
        \onslide<+(1)->{
            \node (f\i) at(\x,-5) {[\(\l\)]};
            \ifnum\p=\pt\draw[-Kite] (e\p) -- (f\i);\else \draw[-Kite] (e\p) -- (f\i);\draw[-Kite] (e\pt) -- (f\i); \fi
            \draw[densely dotted,gray,-Kite] (\px) -- (f\i);
        }
    }

    \foreach[count=\i] \l/\x/\p/\px/\pt in {{1,2,\pivotelem3,4,5}/-6/f1/c1/f2,{\pivotelem7}/-2/c2//c2}{
        \onslide<+(1)->{
            \node (g\i) at(\x,-6) {[\(\l\)]};
            \ifx\p\pt\draw[densely dotted,gray,-Kite] (\p) -- (g\i);\else \draw[-Kite] (\p) -- (g\i);\draw[-Kite] (\pt) -- (g\i); \fi
            \ifx\px\empty\else\draw[densely dotted,gray,-Kite] (\px) -- (g\i);\fi
        }
    }

    \foreach[count=\i] \l/\x/\p/\px/\pt in {{1,2,3,4,5,\pivotelem6,7}/-4/g1/b1/g2,{}/4/b2//b2}{
        \onslide<+(1)->{
            \node (h\i) at(\x,-7) {[\(\l\)]};
            \ifx\p\pt\draw[densely dotted,gray,-Kite] (\p) -- (h\i);\else \draw[-Kite] (\p) -- (h\i);\draw[-Kite] (\pt) -- (h\i); \fi
            \ifx\px\empty\else\draw[densely dotted,gray,-Kite] (\px) -- (h\i);\fi
        }
    }

    \foreach[count=\i] \l/\x/\p/\px/\pt in {{1,2,3,4,5,6,7,\pivotelem8}/0/h1/a/h2}{
        \onslide<+(1)->{
            \node (i\i) at(\x,-8) {[\(\l\)]};
            \ifx\p\pt\draw[densely dotted,gray,-Kite] (\p) -- (i\i);\else \draw[-Kite] (\p) -- (i\i);\draw[-Kite] (\pt) -- (i\i); \fi
            \ifx\px\empty\else\draw[densely dotted,gray,-Kite] (\px) -- (i\i);\fi
        }
    }

    % \onslide<+(1)->{\draw[densely dashed] (-8,-3.5) -- ++(17.25,0) node[above left] {\strut Divide} node[below left] {\strut Merge};}
\end{tikzpicture}
}
    \end{center}
\end{frame}
\fi

\newsavebox\pinguquicksortA \savebox\pinguquicksortA{\PinguBanner{1}{eyes shiny,santa hat}} \newsavebox\pinguquicksortAx \savebox\pinguquicksortAx{\PinguBanner{1}{eyes wink,santa hat}}
\newsavebox\pinguquicksortB \savebox\pinguquicksortB{\PinguBanner{2}{eyes shiny,construction helmet}} \newsavebox\pinguquicksortBx \savebox\pinguquicksortBx{\PinguBanner{2}{eyes wink,construction helmet}}
\newsavebox\pinguquicksortC \savebox\pinguquicksortC{\PinguBanner{3}{eyes shiny,glasses,halo}} \newsavebox\pinguquicksortCx \savebox\pinguquicksortCx{\PinguBanner{3}{eyes wink,glasses,halo}}
\newsavebox\pinguquicksortD \savebox\pinguquicksortD{\PinguBanner{4}{eyes shiny,headband}} \newsavebox\pinguquicksortDx \savebox\pinguquicksortDx{\PinguBanner{4}{eyes wink,headband}}
\newsavebox\pinguquicksortE \savebox\pinguquicksortE{\PinguBanner{5}{eyes shiny,headphones}} \newsavebox\pinguquicksortEx \savebox\pinguquicksortEx{\PinguBanner{5}{eyes wink,headphones}}
\newsavebox\pinguquicksortF \savebox\pinguquicksortF{\PinguBanner{6}{eyes shiny,crown}} \newsavebox\pinguquicksortFx \savebox\pinguquicksortFx{\PinguBanner{6}{eyes wink,crown}}
\newsavebox\pinguquicksortG \savebox\pinguquicksortG{\PinguBanner{7}{right eye shiny,eye patch left}} \newsavebox\pinguquicksortGx \savebox\pinguquicksortGx{\PinguBanner{7}{right eye wink,eye patch left}}
\newsavebox\pinguquicksortH \savebox\pinguquicksortH{\PinguBanner{8}{eyes shiny,hat}} \newsavebox\pinguquicksortHx \savebox\pinguquicksortHx{\PinguBanner{8}{eyes wink,hat}}

\def\pinguqsbx#1{\scalebox{.325}{\expandafter\usebox\csname pinguquicksort#1\endcsname}}
\makeatletter
\newcommand\pinguqsbb[3][]{\protected@edef\tmp{#3}\savebox0{\smash{\pinguqsbx{#2\ifx\tmp\empty\else x\fi}}}\usebox0\llap{\raisebox{-.285cm}{\parbox{\wd0}{#1\small\centering\strut\ifx\tmp\empty\else\faCaretUp\fi\\[-2mm]\T{\strut#3}}}}}
\def\pingubbfn#1{\hfill\parbox{4.5cm}{\footnotesize\color{gray}#1}\bigskip\par}
\def\pingubbtk#1#2{\begin{tikzpicture}[baseline={([yshift=.4cm]1.base)}]
    \def\l{{\color{gray}\scriptsize\faAngleLeft}}\foreach[count=\i] \col/\pin/\t/\m in {#1} {
        \node[inner ysep=0pt,outer ysep=-2pt] (\i) at(1.2*\i,0) {\pinguqsbb[\col]{\pin}{\t}};
        \ifx\m\empty\else
            \fill[white,opacity=.5]([xshift=1mm,yshift=3mm]\i.south west) rectangle ([xshift=-1mm,yshift=1.25cm]\i.north east);
        \fi
    }
    #2
\end{tikzpicture}}%
\makeatother
\begin{frame}[c]{Übungsblatt 10 - Aufgabe 1\hfill Quicksort -- Vorlesungstreue}
\only<2|handout:0>{Ich köööönnte es wieder mit Klammern machen.}
\only<3->{\pingubbtk{\color{paletteA}/H/Pivot/,/F/{\only<5->{\l\,}l}/,/C//,/E//,/A//,/B//,/G//,/D/{\only<5->{\l\,}r}/}{\onslide<5->{\draw[Kite-Kite] (1.south) to[bend right=4] (8.south);}}%\b Fl1\x C\x E\x A\x B \x F\b Dr2%
    \onslide<4->{\pingubbfn{Nach der Vorlesung muss das Pivotelement ans Ende!}}
    \onslide<6->{%
        \pingubbtk{/D/l/,/F//,/C//,/E//,/A//,/B//,/G/r/,\color{paletteA}/H/Pivot/}{\onslide<8->{
            \foreach[remember=\i as \li (initially 1)] \i in {2,...,8}{\draw[-Kite,gray] ([xshift=1pt]\li.south) to[bend right] ([xshift=-1pt]\i.south);}
        }}
    }\onslide<7->{\pingubbfn{\T{l} geht nach rechts, solange es auf ein Element zeigt, das kleiner als das Pivotelement ist.}}
    \onslide<9->{%
        \pingubbtk{/D/l/,/F//,/C//,/E//,/A//,/B//,/G/r/,\color{paletteA}/H/{{\color{black}l,}Pivot}/}{\onslide<11->{\draw[-Kite] (8) to[loop below] (8);}}
    }\onslide<10->{\pingubbfn{Es ist \(l \geq r\). Tausche \T{l} mit dem Pivotelement (hier die selben).}}
}\vspace*{-1.33cm}
\end{frame}

\begin{frame}[c]{Übungsblatt 10 - Aufgabe 1\hfill Quicksort -- Vorlesungstreue}
\onslide<1->{\pingubbtk{\color{paletteA}/D/Pivot/, /F/{\only<3->{\l\,}l}/,/C//,/E//,/A//,/B//,/G/{\only<3->{\l\,}r}/,/H//:ghost}{\onslide<3->{\draw[Kite-Kite] (1.south) to[bend right=4] (7.south);}}}
    \onslide<2->{\pingubbfn{Wir wiederholen den Prozess für die linke und die rechte Hälfte. Hier gibt es nur eine linke\ldots}}
    \onslide<4->{\pingubbtk{/G/l/,/F//,/C//,/E//,/A//,/B/r/,\color{paletteA}/D/Pivot/,/H//:ghost}{\onslide<6->{
        % \foreach[remember=\i as \li (initially 1)] \i in {2,...,6}{\draw[-Kite,gray] ([xshift=1pt]\li.south) to[bend right] ([xshift=-1pt]\i.south);}
        \onslide<9->{\draw[Kite-Kite] (1.south) to[bend right=4] (6.south);}
    }}}
    \onslide<5->{\pingubbfn{Wir tauschen sofort. \(r\) zeigt auf ein Element kleiner und \(l\) auf eines größer gleich dem Pivotelement.}}
    \onslide<7->{\pingubbtk{/B/l/,/F//,/C//,/E//,/A//,/G/r/,\color{paletteA}/D/Pivot/,/H//:ghost}{
        \onslide<9->{\foreach[remember=\i as \li (initially 1)] \i in {2}{\draw[-Kite,gray] ([xshift=1pt]\li.south) to[bend right] ([xshift=-1pt]\i.south);}}
        \onslide<10->{\foreach[remember=\i as \li (initially 5)] \i in {6}{\draw[Kite-,gray] ([xshift=1pt]\li.south) to[bend right] ([xshift=-1pt]\i.south);}}
        \onslide<11->{\draw[Kite-Kite] (2.south) to[bend right=10] (5.south);}
    }}
    \onslide<8->{\pingubbfn{Aaaand we continue with the (little) steppsies. And the swappsies.}}
\end{frame}

\begin{frame}[c]{Übungsblatt 10 - Aufgabe 1\hfill Quicksort -- Vorlesungstreue}
\onslide<1->{\pingubbtk{/B//,/A/l/,/C//,/E//,/F/r/,/G//,\color{paletteA}/D/Pivot/,/H//:ghost}{\onslide<3->{
    \foreach[remember=\i as \li (initially 2)] \i in {3,4}{\draw[-Kite,gray] ([xshift=1pt]\li.south) to[bend right] ([xshift=-1pt]\i.south);}}
    \onslide<4->{\foreach[remember=\i as \li (initially 4)] \i in {5}{\draw[Kite-,gray] ([xshift=1pt]\li.south) to[bend right] ([xshift=-1pt]\i.south);}}
}}
    \onslide<2->{\pingubbfn{Und sie kommen ganz nah zusammen\ldots.}}
    \onslide<5->{\pingubbtk{/B//,/A//,/C//,/E/{l,r}/,/F//,/G//,\color{paletteA}/D/Pivot/,/H//:ghost}{
        \onslide<7->{\draw[Kite-Kite] (4.south) to[bend right=7] (7.south);}
    }}
    \onslide<6->{\pingubbfn{Es ist \(l = r\), wir vertauschen also \(l\) und das Pivotelement.}}
    \onslide<8->{\pingubbtk{\color{paletteA}/B/Pivot/,/A/\only<9->{\l\,}l/,/C/\only<9->{\l\,}r/,/D//:ghost,/F//,/G//,/E//,/H//:ghost}{
        \onslide<9->{\draw[Kite-Kite] (1.south) to[bend right=4] (3.south);}
    }}
    \onslide<9->{\pingubbfn{Meet me at my happy place. We go to the left.}}
\end{frame}

\begin{frame}[c]{Übungsblatt 10 - Aufgabe 1\hfill Quicksort -- Vorlesungstreue}
\onslide<1->{\pingubbtk{/C/l/,/A/r/,\color{paletteA}/B/Pivot/,/D//:ghost,/F//,/G//,/E//,/H//:ghost}{\onslide<3->{\draw[Kite-Kite] (1.south) to[bend right=8] (2.south);}}}
    \onslide<2->{\pingubbfn{Für die linke Hälfte machen wir ein swappsie.}}
    \onslide<4->{\pingubbtk{/A/l/,/C/r/,\color{paletteA}/B/Pivot/,/D//:ghost,/F//,/G//,/E//,/H//:ghost}{\onslide<6->{\foreach[remember=\i as \li (initially 1)] \i in {2}{\draw[-Kite,gray] ([xshift=1pt]\li.south) to[bend right] ([xshift=-1pt]\i.south);}}
    \onslide<7->{\draw[Kite-Kite] (2.south) to[bend right] (3.south);}
    }}
    \onslide<5->{\pingubbfn{Weiter geht \(l\) eins nach rechts, da es nun auf ein kleineres Element zeigt. Nun hat es \(r\) getroffen, wir tauschen also \(l\) und Pivot.}}
    \onslide<8->{\pingubbtk{/A//,/B//:ghost,/C//,/D//:ghost,\color{paletteA}/F/Pivot/,/G/\only<9->{\l\,}l/,/E/\only<9->{\l\,}r/,/H//:ghost}{
        % \onslide<11->{\draw[Kite-Kite] (4.south) to[bend right] (5.south);}
        % \onslide<10->{\foreach[remember=\i as \li (initially 3)] \i in {4}{\draw[-Kite,gray] ([xshift=1pt]\li.south) to[bend right] ([xshift=-1pt]\i.south);}}}
    }}
    \onslide<9->{\pingubbfn{Eigentlich gings rekrusiv weiter. Einelementige oder sortierte Listen lassen sich durch Kommentare überspringen (\say{passiert ja nichts}).}}
\end{frame}

\begin{frame}[c]{Übungsblatt 10 - Aufgabe 1\hfill Quicksort -- Vorlesungstreue}
\onslide<1->{\pingubbtk{/A//,/B//:ghost,/C//,/D//:ghost,\color{paletteA}/F/Pivot/,/G/\only<9->{\l\,}l/,/E/\only<9->{\l\,}r/,/H//:ghost}{\onslide<3->{\draw[Kite-Kite] (5.south) to[bend right=4] (7.south);}}}
    \onslide<2->{\pingubbfn{Es geht also wieder um ein Dreierpärchen}}
    \onslide<4->{\pingubbtk{/A//,/B//:ghost,/C//,/D//:ghost,/E/l/,/G/r/,\color{paletteA}/F/Pivot/,/H//:ghost}{\onslide<6->{\foreach[remember=\i as \li (initially 5)] \i in {6}{\draw[-Kite,gray] ([xshift=1pt]\li.south) to[bend right] ([xshift=-1pt]\i.south);}}\onslide<7->{\draw[Kite-Kite] (6.south) to[bend right] (7.south);}}}
    \onslide<5->{\pingubbfn{Es zeigt \(l\) auf ein kleineres Element, es geht also eins nach rechts, trifft \(r\) und tauscht mit Pivot.}}
    \onslide<8->{\pingubbtk{\color{paletteA}/A/{{\color{black}l,r,}Pivot}/,/B//:ghost,/C//,/D//:ghost,/E//,/F//:ghost,/G//,/H//:ghost}{}}
    \onslide<9->{\pingubbfn{Für alle weiteren Teillisten ist \(l = r = \text{Pivot}\). Wir verkürzen das in dieser Variante}}
\end{frame}

\MakeThePinguExplainIt[text width=4.25cm]{cap=!hide,glasses=!hide,headband,eyes wink,cup=!hide,heart=shadeA,right item angle=-74}{Lesser \& Greater tauschen, bis \(l \geq r\). Dann \(l\) \& Pivot und rekursiv.}
\begin{frame}[c]{Übungsblatt 10 - Aufgabe 1\hfill Quicksort -- Vorlesungstreue}
    \small\begin{enumerate}[<+(1)->]
        \item\label{10inits} Betrachte die gegebene Teilliste \info{initial die Ganze}.
        \item Tausche das Pivotelement ans Ende.
        \item Setze \T{l} ganz nach links und \T{r} ganz nach rechts. \infoblock{Wahlweise auch \T{r} direkt eins links vom Pivotelement, diese Verschiebung passiert immer.}
        \item \label{10mv-l}Schiebe \T{l} nach rechts solange es auf Elemente kleiner als das Pivotelement zeigt \info{\textbf{\paletteA{l}}esser}.
        \item Nur wenn \T{l} noch weiter links als \T{r} ist, schiebe \T{r} nach links, solange es auf Elemente größer gleich dem Pivotelement zeigt \info{g\textbf{\paletteA{r}}eater}.
        \item Ist \T{l} nun immer noch weiter links von \T{r}, dann vertausche die Elemente auf die sie zeigen und wiederhole bei \ref{10mv-l}. Andernfalls, vertausche \T{l} und Pivotelement.
        \item Wiederhole ab \ref{10inits} rekursiv für die Teilliste links und rechts vom Pivotelement.
    \end{enumerate}
\begin{tikzpicture}[overlay,remember picture]
    \onslide<9->{\node[left=-7mm,scale=.8] at(current page.9) {\usebox\pinguexplainbox};}
\end{tikzpicture}
\end{frame}


% list | left | right
\def\shortqspivot#1#2#3{\foreach[count=\i] \l in {#1} {\node (\i) at (.5*\i,0) {\strut\l};}
\ifnum#2>0 \ifnum#2=#3
\node[above=-2.75mm,align=center] (l) at (#2.north) {l,r\\[-2.33mm]\footnotesize\faCaretDown};
\else\node[above=-2.75mm,align=center] (l) at (#2.north) {l\\[-2.33mm]\footnotesize\faCaretDown}; \node[above=-2.75mm,align=center] (r) at (#3.north) {r\\[-2.33mm]\footnotesize\faCaretDown};\fi\fi}
{\def\leftc#1{\parbox{.45\linewidth}{\raggedleft\scriptsize\color{gray}#1}~~}
\def\rightc#1{~~\parbox{.45\linewidth}{\raggedright\scriptsize\color{gray}#1}}
\begin{frame}[c]{Übungsblatt 10 - Aufgabe 1\hfill Quicksort -- Nur mit Markern}
\begin{itemize}[<+(1)->]
    \item Von Hand, müssen wir aber das Pivotelement nirgends \say{hintauschen}.
    \item Im Folgenden eine Kompaktversion. Mit den Pivotelementen \pivotelem{P}:
\end{itemize}
\begin{columns}[c,onlytextwidth]
\column{.5\linewidth}
\centering\onslide<5->{\leftc{1) Solange \(a[r] \geq P\) und \(a[l] < P\)}}\scalebox{.85}{\begin{tikzpicture}[align-base]
    \onslide<4->{\shortqspivot{\pivotelem{8},6,3,5,1,2,7,4}{2}{8}}
    \onslide<6->{\foreach[remember=\i as \li (initially 2)] \i in {3,...,8}{\draw[-Kite,gray] ([xshift=3pt,yshift=7pt]\li.north) -- ([xshift=-3pt,yshift=7pt]\i.north);}}
\end{tikzpicture}}\par
\onslide<8->{\leftc{2) Es ist: \(l \geq r\). Hier tauschen wir \T{r} und \T{p}}}\scalebox{.85}{\begin{tikzpicture}[align-base]
    \onslide<7->{\shortqspivot{\pivotelem{8},6,3,5,1,2,7,4}{8}{8}}
    \onslide<9->{\draw[Kite-Kite] ([yshift=5pt]1.south) to[bend right=10] ([yshift=5pt]8.south);}
\end{tikzpicture}}\par
\onslide<11->{\leftc{3) Rekursion links. Solange \(a[r] \geq P\) und \(a[l] < P\)}}\scalebox{.85}{\begin{tikzpicture}[align-base]
    \onslide<10->{\shortqspivot{\pivotelem{4},6,3,5,1,2,7,{\color{gray}8}}{2}{7}}
    \onslide<12->{\foreach[remember=\i as \li (initially 7)] \i in {6}{\draw[-Kite,gray] ([xshift=-3pt,yshift=7pt]\li.north) -- ([xshift=3pt,yshift=7pt]\i.north);}}
\end{tikzpicture}}\par
\onslide<14->{\leftc{4) \(a[l] \geq P\) und \(a[r] < P\)}}\scalebox{.85}{\begin{tikzpicture}
    \onslide<13->{\shortqspivot{\pivotelem{4},6,3,5,1,2,7,{\color{gray}8}}{2}{6}}
    \onslide<15->{\draw[Kite-Kite] ([yshift=5pt]2.south) to[bend right=10] ([yshift=5pt]6.south);}
\end{tikzpicture}}
\column{.5\linewidth}
\scalebox{.85}{\begin{tikzpicture}[align-base]
    \onslide<16->{\shortqspivot{\pivotelem{4},2,3,5,1,6,7,{\color{gray}8}}{2}{6}}
    \onslide<19->{\foreach[remember=\i as \li (initially 2)] \i in {3,...,4}{\draw[-Kite,gray] ([xshift=3pt,yshift=7pt]\li.north) -- ([xshift=-3pt,yshift=7pt]\i.north);}}
    \onslide<18->{\foreach[remember=\i as \li (initially 6)] \i in {5}{\draw[-Kite,gray] ([xshift=-3pt,yshift=7pt]\li.north) -- ([xshift=3pt,yshift=7pt]\i.north);}}
\end{tikzpicture}}\onslide<17->{\rightc{5) Solange \(a[r] \geq P\) und \(a[l] < P\)}}\par\scalebox{.85}{\begin{tikzpicture}[align-base]
    \onslide<20->{\shortqspivot{\pivotelem{4},2,3,5,1,6,7,{\color{gray}8}}{4}{5}}
    \onslide<22->{\draw[Kite-Kite] ([yshift=5pt]4.south) to[bend right=10] ([yshift=5pt]5.south);
    \draw[Kite-Kite,white] ([yshift=5pt]1.south) to[bend right=10] ([yshift=5pt]8.south);}
\end{tikzpicture}}\onslide<21->{\rightc{6) \(a[l] \geq P\) und \(a[r] < P\)}}\par\scalebox{.85}{\begin{tikzpicture}[align-base]
    \onslide<23->{\shortqspivot{\pivotelem{4},2,3,1,5,6,7,{\color{gray}8}}{4}{5}}
    \onslide<25->{\foreach[remember=\i as \li (initially 5)] \i in {4}{\draw[-Kite,gray] ([xshift=-3pt,yshift=7pt]\li.north) -- ([xshift=3pt,yshift=7pt]\i.north);}}
\end{tikzpicture}}\onslide<24->{\rightc{7) Solange \(a[r] \geq P\) und \(a[l] < P\) (hier nicht, da \(l \geq r\).)}}\par\scalebox{.85}{\begin{tikzpicture}[align-base]
    \onslide<26->{\shortqspivot{\pivotelem{4},2,3,1,5,6,7,{\color{gray}8}}{4}{4}}
    \onslide<28->{\draw[Kite-Kite,white] ([yshift=5pt]2.south) to[bend right=10] ([yshift=5pt]6.south);
    \draw[Kite-Kite] ([yshift=5pt]1.south) to[bend right=10] ([yshift=5pt]4.south);}
\end{tikzpicture}}\onslide<27->{\rightc{8) Es ist \(l \geq r\).}}
\end{columns}
\begin{center}
    \begin{tikzpicture}
        \onslide<29->{\shortqspivot{1,2,3,{\color{gray}4},5,6,7,{\color{gray}8}}{0}{0}}
    \end{tikzpicture}\vspace*{-.9\baselineskip}\par
    \onslide<30->{{\scriptsize\color{gray}Fertig sortiert, alle weiteren Anwendungen liefern nichts.}}
\end{center}
\end{frame}}

{\def\sorted#1#2#3{\draw[decorate,decoration={brace,amplitude=6pt,mirror}] ([yshift=-8pt,xshift=-5pt]#1.south west) to[edge node={node[below=6pt,overlay] {#3}}] ([yshift=-8pt,xshift=5pt]#2.south east);}
\begin{frame}[c]{Übungsblatt 10 - Aufgabe 1\hfill Insertionsort}
\centering
\scalebox{.85}{\begin{tikzpicture}
    \onslide<2->{
        \foreach[count=\i] \x in {8,6,3,5,1,2,7,4} {\node[dot]  (\x) at(1.5*\i,0) {\x};}
    }
    \onslide<3->{\sorted{6}{4}{unsortiert} \sorted{8}{8}{sortiert} }
    \pgfinterruptboundingbox
    \onslide<4->{\draw[-Kite] (6.north) to[bend right=22] ([xshift=-.66cm]8.north);}
    \endpgfinterruptboundingbox
\end{tikzpicture}}\\[2.75em]
\scalebox{.85}{\begin{tikzpicture}
    \onslide<5->{
        \foreach[count=\i] \x in {6,8,3,5,1,2,7,4} {\node[dot]  (\x) at(1.5*\i,0) {\x};}
    }
    \onslide<6->{\sorted{3}{4}{unsortiert} \sorted{6}{8}{sortiert} }
    \pgfinterruptboundingbox
    \onslide<7->{\draw[-Kite] (3.north) to[bend right=18] ([xshift=-.66cm]6.north);}
    \endpgfinterruptboundingbox
\end{tikzpicture}}\\[2.75em]
\scalebox{.85}{\begin{tikzpicture}
    \onslide<8->{
        \foreach[count=\i] \x in {3,6,8,5,1,2,7,4} {\node[dot]  (\x) at(1.5*\i,0) {\x};}
    }
    \onslide<9->{\sorted{5}{4}{unsortiert} \sorted{3}{8}{sortiert} }
    \pgfinterruptboundingbox
    \onslide<10->{\draw[-Kite] (5.north) to[bend right=18] ([xshift=-.66cm]6.north);}
    \endpgfinterruptboundingbox
\end{tikzpicture}}
\end{frame}

\begin{frame}[c]{Übungsblatt 10 - Aufgabe 1\hfill Insertionsort, II}
\centering
\scalebox{.85}{\begin{tikzpicture}
    \onslide<2->{
        \foreach[count=\i] \x in {3,5,6,8,1,2,7,4} {\node[dot]  (\x) at(1.5*\i,0) {\x};}
    }
    \onslide<3->{\sorted{1}{4}{unsortiert} \sorted{3}{8}{sortiert}}
    \pgfinterruptboundingbox
    \onslide<4->{\draw[-Kite] (1.north) to[bend right=12] ([xshift=-.66cm]3.north);}
    \endpgfinterruptboundingbox
\end{tikzpicture}}\\[2.75em]
\scalebox{.85}{\begin{tikzpicture}
    \onslide<5->{
        \foreach[count=\i] \x in {1,3,5,6,8,2,7,4} {\node[dot]  (\x) at(1.5*\i,0) {\x};}
    }
    \onslide<6->{\sorted{2}{4}{unsortiert} \sorted{1}{8}{sortiert}}
    \pgfinterruptboundingbox
    \onslide<7->{\draw[-Kite] (2.north) to[bend right=12] ([xshift=-.66cm]3.north);}
    \endpgfinterruptboundingbox
\end{tikzpicture}}\\[2.75em]
\scalebox{.85}{\begin{tikzpicture}
    \onslide<8->{
        \foreach[count=\i] \x in {1,2,3,5,6,8,7,4} {\node[dot]  (\x) at(1.5*\i,0) {\x};}
    }
    \onslide<9->{\sorted{7}{4}{unsortiert} \sorted{1}{8}{sortiert}}
    \pgfinterruptboundingbox
    \onslide<10->{\draw[-Kite] (7.north) to[bend right=15] ([xshift=-.66cm]8.north);}
    \endpgfinterruptboundingbox
\end{tikzpicture}}
\end{frame}

\begin{frame}[c]{Übungsblatt 10 - Aufgabe 1\hfill Insertionsort, II}
\centering
\scalebox{.85}{\begin{tikzpicture}
    \onslide<2->{
        \foreach[count=\i] \x in {1,2,3,5,6,7,8,4} {\node[dot]  (\x) at(1.5*\i,0) {\x};}
    }
    \onslide<3->{\sorted{4}{4}{unsortiert} \sorted{1}{8}{sortiert}}
    \pgfinterruptboundingbox
    \onslide<4->{\draw[-Kite] (4.north) to[bend right=12] ([xshift=-.66cm]5.north);}
    \endpgfinterruptboundingbox
\end{tikzpicture}}\\[2.75em]
\scalebox{.85}{\begin{tikzpicture}
    \onslide<5->{
        \foreach[count=\i] \x in {1,2,3,4,5,6,7,8} {\node[dot]  (\x) at(1.5*\i,0) {\x};}
    }
    \onslide<6->{\sorted{1}{8}{sortiert}}
\end{tikzpicture}}\\[2.25em]
\begin{itemize}
    \item<7-> Theoretisch kann man auch alle \say{Zwischentauschschritte} einzeichnen.
\end{itemize}
\end{frame}}

\subsection{Aufgabe 2}
\iffull
\begin{frame}[c]{Übungsblatt 10 - Aufgabe 2}
    \onslide<2->{In dieser Aufgabe sollen Sie den \textit{Bubblesort} Algorithmus implementieren.
    Ihre Implementierung soll die Wrapper-Methode \bjava{public static void bubbleSort(int[]} \bjava{array, int n)} anbieten, welche das zu sortierende Array und dessen Länge als Parameter übernimmt.
    Implementieren Sie nun zusätzlich eine rekursive Hilfsmethode mit geeigneten Parametern, die den \textit{Bubblesort} Algorithmus ausführt.\medskip\par
    Verwenden Sie in ihrer Lösung keine Schleifen und auch keine vorgefertigten Sortierfunktionen wie z.B. \texttt{Arrays.sort()}!}\bigskip

    \onslide<3->{\resizebox{2cm}!{\tikz[align-half-base]\pingu[eyes wink,left wing wave, sign post left={Note!}];}\hfill\parbox{\linewidth-2cm}{%
    Wir betrachten einmal eine naive Implementation von Bubblesort und eine optimierte Version, welche eine Abbruchbedingung benutzt.}}
\end{frame}
\fi

\MakeThePinguExplainIt[text width=8.75cm]{cap=!hide,strawhat,glasses=!hide,glasses round,eyes shiny,cup=!hide,heart=shadeA,gold medal,right item angle=5}{Florians Didaktik Hoffnungsversuch: Eigentlich machen wir Rekursion nicht \textit{nur um irgendwas iteratives hinzubekommen}. Es gibt schöne, wundervolle Probleme die sich mit Rekursion wunderschön und elegant lösen lassen. Tolle\ldots\ wirklich tolle\ldots\ schnief.}
\begin{frame}[c,fragile]{Übungsblatt 10 - Aufgabe 2}
    \begin{itemize}[<+(1)->]
        \item Iterativ ist das einfach:\lstfs{9}
\begin{plainjava}
!*\onslide<3->*!public static void bubbleSort(int[] arr) {
!*\onslide<4->*!    for (int i = arr.length; i > 1; i--) {
!*\onslide<5->*!        for (int j = 0; j < i - 1; j++) {
!*\onslide<6->*!            if (arr[j] > arr[j + 1]) { // swap(arr, j, j + 1)
!*\onslide<7->*!                int tmp = arr[j];
!*\onslide<7->*!                arr[j] = arr[j + 1];
!*\onslide<7->*!                arr[j + 1] = tmp;
!*\onslide<6->*!            }
!*\onslide<5->*!        }
!*\onslide<4->*!    }
!*\onslide<3->*!}
\end{plainjava}
    \item<8-> Wir werden zweimal Rekursion benutzen um die For-Schleifen zu re-\\präsentieren: \only<8->{\textattachfile{\curpath BubbleSort.java}{BubbleSort.java}}.
\end{itemize}
\begin{tikzpicture}[overlay,remember picture]
\onslide<9->{\node[left=-7mm,scale=.8] at(current page.-15) {\usebox\pinguexplainbox};}
\begin{uncoverenv}<10->\node[below left,yshift=-14.7mm,text width=9.25cm,scale=.575] at(current page.north east) {
\begin{plainjava}
static void swap(int[] array, int i, int j) {
    int tmp = array[i];
    array[i] = array[j];
    array[j] = tmp;
}
\end{plainjava}};
\end{uncoverenv}
\end{tikzpicture}
\end{frame}

\begin{frame}[c,fragile]{Übungsblatt 10 - Aufgabe 2}
\begin{plainjava}
!*\onslide<2->*!public static void bubbleSort(int[] array, int n) {
!*\onslide<3->*!    if (n > 1) {
!*\onslide<5->*!        move(array, 0, n);
!*\onslide<4->*!        bubbleSort(array, n - 1);
!*\onslide<3->*!    }
!*\onslide<2->*!}
!*\onslide<6->*!private static void move(int[] array, int i, int n) {
!*\onslide<7->*!    if (i < n - 1) {
!*\onslide<9->*!        if (array[i] > array[i + 1])
!*\onslide<9->*!            swap(array, i, i + 1);
!*\onslide<8->*!        move(array, i + 1, n);
!*\onslide<7->*!    }
!*\onslide<6->*!}!*\onslide<1->*!
\end{plainjava}
\begin{tikzpicture}[overlay,remember picture]
\begin{uncoverenv}<1->\node[above left,yshift=5.7mm,text width=9.25cm,scale=.575] at(current page.south east) {
\begin{plainjava}
public static void bubbleSort(int[] arr) {
    for (int n = arr.length; n > 1; n--) {
        for (int i = 0; i < n - 1; i++) {
            if (arr[i] > arr[i + 1])
                swap(arr, i, i + 1);
        }
    }
}
\end{plainjava}};
\end{uncoverenv}
\end{tikzpicture}
\end{frame}

\iffull
\begin{frame}[c,fragile]{Übungsblatt 10 - Aufgabe 2}
\SetupLstHl\begin{itemize}[<+(1)->]
    \item Wir fügen eine Variable hinzu, ob getauscht wurde (\only<2->{\textattachfile{\curpath OptimizedBubbleSort.java}{OptimizedBubbleSort.java}}):
\lstfs{9}
\begin{plainjava}
!*\onslide<3->*!|ihl|public static void bubbleSort(int[] array, int n) {|ihl|
!*\onslide<4->*!    |ihl|if (n > 1) {|ihl|
!*\onslide<6->*!        if(!|ihl|move(array, 0, n,|ihl| false|ihl|))|ihl| return; // kein Tausch mehr?
!*\onslide<5->*!        |ihl|bubbleSort(array, n - 1);|ihl|
!*\onslide<4->*!    |ihl|}|ihl|
!*\onslide<3->*!|ihl|}|ihl|
!*\onslide<7->*!|ihl|private static boolean move(int[] array, int i, int n,|ihl| boolean swapped|ihl|) {|ihl|
!*\onslide<8->*!    |ihl|if(i < n - 1) {|ihl|
!*\onslide<11->*!        |ihl|if (array[i] > array[i+1])|ihl| {
!*\onslide<12->*!            |ihl|swap(array, i, i + 1);|ihl|
!*\onslide<13->*!            swapped = true;
!*\onslide<11->*!        }
!*\onslide<9->*!        |ihl|return move(array, i + 1, n,|ihl| swapped|ihl|);|ihl|
!*\onslide<8->*!    |ihl|}|ihl|
!*\onslide<10->*!    return swapped;
!*\onslide<7->*!}
\end{plainjava}
\end{itemize}
\end{frame}
\fi

\subsection{Aufgabe 3}
\begin{frame}[c,fragile]{Übungsblatt 10 - Aufgabe 3}
    \begin{enumerate}[<+(1)->]
        \itemsep7pt
        \item Was versteht man unter einem stabilen Sortieralgorithmus?\medskip\par
            \pause{} Ein \textit{stabiler} Sortieralgorithmus verändert nicht die Reihenfolge von Elementen mit dem gleichen \textit{Sortierschlüssel}. \infoblock{Beispiel: Wir sortieren zuerst nach Alter, danach nach Alphabet. Ein stabiler Sortieralgorithmus garantiert, dass Personen mit gleichem Alter immernoch alphabetisch sortiert sind.}
        \item Welche der vier auf diesem Übungsblatt vorgekommenen Sortieralgorithmen sind stabil und welche sind instabil?\medskip\par
        \pause Insertionsort, Mergesort und Bubblesort sind stabil, Quicksort ist instabil.\pause{} \infoblock{Genau genommen kann man aber jeden Algorithmus mit entsprechendem Zusatzaufwand stabil und nicht stabil implementieren. Wir halten uns hier an die Vorlesungsimplementation.}
    \end{enumerate}
\end{frame}

\subsection{Zusatzaufgabe 4/6}

\iffull
\begin{frame}[c,fragile]{Übungsblatt 10 - Zusatzaufgabe 4/6}
    \onslide<2->{\footnotesize In dieser Aufgabe sollen Sie sich mit dem 0-1-Rucksackproblem beschäftigen. Hier gibt es einen Rucksack mit einer maximalen Tragekapazität $W_{max}$, sowie $n$ Gegenstände welche jeweils ein Gewicht $w_i$ und einen Wert $v_i$ haben. Die gesuchte Lösung ist der maximale Wert $V$ einer Teilmenge von Gegenständen, deren Gesamtgewicht nicht größer als $W_{max}$ ist.

    Das Problem lässt sich durch folgende rekursive Formel definieren:
    \[
    \text{k}(W_{max},i) =
    \left\{
        \begin{array}{ll}
            0  & \mbox{für } i > n \\
            \text{k}(W_{max},i+1)  & \mbox{für } w_i > W_{max} \\
            max\bigl(\text{k}(W_{max},i+1), \text{k}(W_{max} - w_i,i+1) + v_i\bigr) & \mbox{sonst }
        \end{array}
    \right.
    \]
    Implementieren Sie die Methode \bjava{public static int knapsack(int W, int[] weights,} \bjava{int[] values)} welche die maximale Tragekapazität des Rucksacks, sowie zwei Integer Arrays für die Gewichte und Werte der Gegenstände als Parameter übernimmt, und den maximalen Wert der Gegenstände zurückgibt die im Rucksack transportiert werden können.
    Implementieren Sie dazu eine rekursive Hilfsmethode mit geeigneten Parametern, welche intern von der Wrapper-Methode aufgerufen wird.\par}
\end{frame}
\fi

\MakeThePinguExplainIt[text width=6.65cm]{cap=!hide,glasses=!hide,construction helmet,eyes shiny,bill flat,right item angle=5}{Wir nutzen \bjava{w.length} für \bjava{n} und müssen beachten, dass es bei \(0\) losgeht.}
\begin{frame}[fragile]{Zusatzaufgabe 4/6}
\begin{itemize}[<+(1)->]
    \item Das Gute an einer rekursiven Definition wie dieser? Wir können Sie direkt umsetzen!
\end{itemize}\SetupLstHl
\begin{plainjava}
!*\onslide<3->*!public static int knapsack(int W, int[] weights, int[] values) {
!*\onslide<4->*!    if(weights.length != values.length) return -1;
!*\onslide<5->*!    return k(W, 0, |ihl|weights, values|ihl|);
!*\onslide<3->*!}
!*\onslide<3->*!
!*\onslide<6->*!private static int k(int W, int i, |ihl|int[] w, int[] v|ihl|) {
!*\onslide<7->*!  if (i >= w.length) return 0; // beginnt bei 0:    !*\solGet{comments}{$i > n$}*! :yields: 0
!*\onslide<8->*!  if (w[i] > W) return k(W, i + 1, |ihl|w, v|ihl|); // !*\solGet{comments}{$w_i > W_{\text{max}}$}*! :yields: !*\solGet{comments}{$k(W_{\text{max}}, i + 1)$}*!
!*\onslide<9->*!  else return Math.max(k(W, i + 1, |ihl|w, v|ihl|), // sonst: !*\solGet{comments}{$\max\bigl(\text{\say{ohne}},$}*!
!*\onslide<9->*!    k(W - w[i], i + 1, |ihl|w, v|ihl|) + v[i]); // !*\solGet{comments}{$\text{\say{mit}}\bigr)$}*!
!*\onslide<6->*!}
\end{plainjava}
\begin{tikzpicture}[overlay,remember picture]
    \onslide<10->{\node[left=-7mm,scale=.8] at(current page.{-6.25}) {\usebox\pinguexplainbox};}
    \onslide<11->{\node[above left,yshift=.5cm,xshift=-.35cm] at(current page.south east) {\only<11->{\textattachfile{\curpath Knapsack.java}{Knapsack.java}}};}
\end{tikzpicture}
\end{frame}

\section{Die End-Pingu-Ente}
\begin{frame}
    \begin{center}
    \begin{tikzpicture}[scale=2.5]
    % \shuffleducks
         \duck[aodai=cprimary!50!black,tshirt=cprimary,conicalhat=brown,water=paletteA]
    \end{tikzpicture}\\\pause{}
    Wieder kein Pingu,\pause{} weil süß!
    \end{center}
\begin{tikzpicture}[overlay, remember picture]
    \fill[black] (current page.north east) -| ++(-2.5cm,-1.9cm) -- ++(0.25cm,-0.25cm) -| cycle;
    \node[below left=2pt,xshift=-.15mm,scale=0.36] at(current page.north east) {\usebox\donotmesswithbeate};
\end{tikzpicture}
\end{frame}

\iffull\end{document}\fi
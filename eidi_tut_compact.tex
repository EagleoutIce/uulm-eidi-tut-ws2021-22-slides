\documentclass[aspectratio=169,ngerman,handout,t]{beamer}
\def\thpath{.}
\input{global.src}
\usepackage{fmtcount}
\DefinePalette{Compact}
{Hellblau,hellbläulich: RGB(21, 110, 130)}
{Türkis,türkisfarben: RGB(21, 128, 112)}
{Blau,bläulich: RGB(21, 92, 148)}
{Grün,grünlich: RGB(21, 150, 90)}
\UsePalette{Compact}
\colorlet{MaterialHeaderColor}{paletteB!75!white}%
\colorlet{NextMaterialHeaderColor}{paletteB!65!white}%
\attachfilesetup{color=paletteB}
\fullfalse
\def\maxtut{8}
\title[Alle Tutorien 0--\maxtut]{Kompaktversion Tutorien\\\small Von pinguinreduzierten Tutorien 0--\maxtut}
\date{\sffamily\today}

% TODO: fix undefined labels

\let\oldinputif\InputIfFileExists

\makeatletter
% sub | title | path | file | num | KW
\newcommand\Load[5][]{%
\section{\protect\strut#2}
\ifx!\detokenize{#1}!\else\def\beamer@shorttitle{#1}\fi
{\setbeamercolor{background canvas}{bg=paletteB}
\begin{frame}[c,plain]{}
   \begin{center}
      \vspace*{4.25mm}\par
      \huge\bfseries\color{pingu@white}
         #2\\[1.25mm]
         \smash{\normalsize\color{paletteB!68!white}KW #5}
   \end{center}
\end{frame}
}
\bgroup\let\section\subsection
\let\subsection\subsubsection
% sub sub stays subsub?
% \let\subsubsection\paragraph
\gappto\input@path{{#3/}}%
\def\curpath{#3/}% for rBash and rExecute
% we block global.src form being loaded
\def\InputIfFileExists##1##2##3{\relax \global\let\InputIfFileExists\oldinputif}
% forbid additional loading:
\renewcommand*\usepackage[2][]{\typeout{gobble: ##1|##2}}
\renewcommand*\RequirePackage[2][]{\typeout{gobble: ##1|##2}}
% get the date:
\typeout{Loading: #3/source_folien_#4.src}
\@input{source_folien_#4.src}\egroup
}
% update prepath
\setbeamercolor{section in toc}{fg=darkgray}

\begin{document}
\Titlepage{0--\maxtut}
\setcounter{tocdepth}{1}
{\setbeamercolor{background canvas}{bg=paletteB}
\def\ImpT#1{\textit{\color{pingu@white}#1}}
\begin{frame}[c,plain]{}
   \begin{center}
      \vspace*{5.25mm}\par
      \bfseries\color{paletteB!68!white}
         Diese \ImpT{Kompaktversion} ist dazu gedacht die wichtigsten Kommentare und Lösungen zu sammeln.
         Sie ist allerdings \ImpT{ohne jede Garantie auf Vollständigkeit} aufzufassen. Zusätzliche Inhalte finden sich in den Folien zu den einzelnen Tutorien.\bigskip\par
         Liebe Grüße, Flo
   \end{center}
\end{frame}
}
\section*{Übersicht}
\pdfbookmark[0]{Übersicht}{eidi@overview}
\setbeamertemplate{section in toc}{%
  {$\HEXADecimalnum{\inserttocsectionnumber}$}.~~\inserttocsection}
\begin{frame}[c,plain]{Eine Sammlung~--- schnief~--- ohne Liebe}
% \def\inserttocsectionnumber{\HEXADecimal{section}.}%
   \begin{multicols}{2}
\disablehyper\tableofcontents
   \end{multicols}
\end{frame}
\setcounter{tocdepth}{5}
\Load[Organisatiorisches Tutorium]{Organisatiorisches Tutorium}{Org-Tutorium}{org}{43}
\Load[Tutorium 0]{Blatt 0 --- Hello World}{0-Tutorium}{0}{43}
\Load[Tutorium 1]{Blatt 1 --- Zahlen \& Algorithmen}{1-Tutorium}{1}{44}
\Load[Tutorium 2]{Blatt 2 --- Raten, Typen \& Booleans}{2-Tutorium}{2}{45}
\Load[Tutorium 3]{Blatt 3 --- Programmfluss \& Schleifen}{3-Tutorium}{3}{46}
\Load[Tutorium 4]{Blatt 4 --- Schleifen, Vokale \& Prim}{4-Tutorium}{4}{47}
\Load[Tutorium 5]{Blatt 5 --- Arrays, Matrizen \& Vektoren}{5-Tutorium}{5}{48}
\Load[Tutorium 6]{Blatt 6 --- Methoden, Tic-Tac-Toe}{6-Tutorium}{6}{49}
\Load[Tutorium 7]{Blatt 7 --- Überladen \& Überschatten}{7-Tutorium}{7}{50}
\Load[Tutorium 8]{Blatt 8 --- Klassen \& Taxis}{8-Tutorium}{8}{2}
\end{document}